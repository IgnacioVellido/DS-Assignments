\section{Contenedor para actividades de ciencia de datos basado en R}

\subsection{Descripción}

Contenedor docker partiendo de una instalación base de R al que se le añaden distintos paquetes de ciencia de datos.

\subsection{Archivo Dockerfile}


% MEJOR CAPTURA

% \begin{lstlisting}
% FROM rocker/r-base
% LABEL author="Ignacio Vellido Expósito"
% ENV http_proxy http://stargate.ugr.es:3128

% RUN apt-get update

% # Se podrían separar para evitar que si uno falla se corte
% # la ejecución. Como está testeado se deja resumido
% RUN R -e "install.packages(c('tidyverse','caret','RSNNS',
%                              'frbs','FSinR','forecast'),
%                             dependencies=TRUE, 
%                             repos='http://cran.rstudio.com/')"

% # Para automatizar el proceso de testeo, se lanza desde aquí
% COPY testDocker.R /home/testDocker.R

% # Lanzar script
% CMD R -e "source('/home/testDocker.R')"
% # CMD "R /home/testDocker.R > /home/testOutput.txt"
% \end{lstlisting}

\subsection{Proceso de construcción}

% Captura docker construído

% Captura docker en funcionamiento

% Captura desde dentro del docker

\subsection{Evaluación}

Para evaluar el correcto funcionamiento se lanza el siguiente script, que carga las librerías instaladas y realiza operaciones con algunas de ellas