\section{Análisis de Centralidad}

\begin{figure}[ht]
    \centerfloat
    \includegraphics[width=0.9\textwidth]{img/resultados/distanciaGrafo/Betweenness Centrality Distribution.png}
    \caption{Intermediación.}
\end{figure}

\begin{figure}[ht]
    \centerfloat
    \includegraphics[width=0.9\textwidth]{img/resultados/distanciaGrafo/Closeness Centrality Distribution.png}
    \caption{Cercanía.}
\end{figure}

\begin{figure}[ht]
    \centerfloat
    \includegraphics[width=0.9\textwidth]{img/resultados/distanciaGrafo/Eccentricity Distribution.png}
    \caption{Excentricidad.}
\end{figure}

\begin{figure}[ht]
    \centerfloat
    \includegraphics[width=0.9\textwidth]{img/resultados/distanciaGrafo/Harmonic Closeness Centrality Distribution.png}
    \caption{Cercanía harmónica.}
\end{figure}

\begin{figure}[t]
    \centering
    \resizebox{0.9\columnwidth}{!}{%
    \begin{tabular}{| l | l | l | l |} 
        \hline
        \textbf{Centralidad de Grado} & \textbf{Intermediación} & \textbf{Cercanía} & \textbf{Vector propio} \\
        \Xhline{2\arrayrulewidth}
        \textbf{7237} - 216 & 	\textbf{7199} - 2,612,617 & \textbf{7199} - 0.2907 & 	\textbf{7237} - 1.0000  \\
        \hline
        \textbf{3530} - 175 & 	\textbf{7237} - 2,486,453 & 	\textbf{7237} - 0.2856 & \textbf{3240} - 0.7149  \\
        \hline
        \textbf{4785} - 174 & 	\textbf{2854} - 2,253,302 & 	\textbf{4356} - 0.2816 & 	\textbf{3597} - 0.7052  \\
        \hline
        \textbf{524}   - 172 & 	\textbf{4356} - 1,953,690 & 	\textbf{2854} - 0.2803 & 	\textbf{763}   - 0.6555  \\
        \hline
        \textbf{3450} - 159 & \textbf{6101} - 1,504,994 & 	\textbf{5454} - 0.2798 & 	\textbf{2083} - 0.5940  \\
        \hline
    \end{tabular}
    }
    \caption{Tabla de actores más relevantes.}
\end{figure}

\begin{figure}[ht]
    \centerfloat
    \includegraphics[width=0.9\textwidth]{img/resultados/grado-vector7237.png}
    \caption{Vecinos del nodo 7237. A mayor grado mayor tamaño, más rojo mayor valor de vector propio.}
\end{figure}

\begin{figure}[ht]
    \centerfloat
    \includegraphics[width=0.9\textwidth]{img/resultados/grado-vector7237-prof2.png}
    \caption{Vecinos del nodo 7237 a profundidad 2.}
\end{figure}

\begin{figure}[ht]
    \centerfloat
    \includegraphics[width=0.9\textwidth]{img/resultados/grado-vector7199-prof2.png}
    \caption{Vecinos del nodo 7199 a profundidad 2.}
\end{figure}

\begin{figure}[ht]
    \centerfloat
    \includegraphics[width=0.9\textwidth]{img/resultados/grado-vector7199y7237.png}
    \caption{Unión de los vecinos del nodo 7237 y del 7199 a profundidad 2.}
\end{figure}

Los valores de la Figura 10 carecen de significado si no nos fijamos en sus gráficas de distribución.

Respecto a la centralidad, como se comentó anteriormente estos outliers es muy probable que correspondan a creadores de contenido, ya que sus valores se encuentran extremadamente separados de la media en la red.

Hacemos notar que los nodos con buena alta intermediación tienen también los valores más altos de cercanía. No nos queda del todo claro cuál puede ser el motivo tras ello, pero es probable que estos nodos estén áltamente conectados con dos o más hubs en la red, tal y como muestra la  Figura ?? % (13, vecinos de 7199)

\vspace{\baselineskip}

Nos vamos a fijar ahora en los dos más importantes.

En la Figura ? se muestran los vecinos a profundidad uno y dos del nodo 7237.
También se muestran los de profundidad dos del nodo 7199, vemos que la zona de la red que ocupan es similar, siendo muy probable que estén conectados ??. Pese a ello las zonas con las que se conectan son diferentes, 

En la Figura ? vemos la unión de los vecinos a profundidad dos de este par de actores tan relevante. Destacamos dos cosas de esta figura: por un lado, la gran importancia de estos actores puesto que con un máximo de dos enlaces alcanzamos un 26.3\% de la red (2043 nodos), y por otra parte la influencia en el flujo de información ya que sabemos que \textbf{en media} añadiéndole solo cuatro enlaces más (para un total de seis) podemos transmitir a toda la red.