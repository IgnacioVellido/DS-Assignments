% Options for packages loaded elsewhere
\PassOptionsToPackage{unicode}{hyperref}
\PassOptionsToPackage{hyphens}{url}
%
\documentclass[
]{article}
\usepackage{lmodern}
\usepackage{amssymb,amsmath}
\usepackage{ifxetex,ifluatex}
\ifnum 0\ifxetex 1\fi\ifluatex 1\fi=0 % if pdftex
  \usepackage[T1]{fontenc}
  \usepackage[utf8]{inputenc}
  \usepackage{textcomp} % provide euro and other symbols
\else % if luatex or xetex
  \usepackage{unicode-math}
  \defaultfontfeatures{Scale=MatchLowercase}
  \defaultfontfeatures[\rmfamily]{Ligatures=TeX,Scale=1}
\fi
% Use upquote if available, for straight quotes in verbatim environments
\IfFileExists{upquote.sty}{\usepackage{upquote}}{}
\IfFileExists{microtype.sty}{% use microtype if available
  \usepackage[]{microtype}
  \UseMicrotypeSet[protrusion]{basicmath} % disable protrusion for tt fonts
}{}
\makeatletter
\@ifundefined{KOMAClassName}{% if non-KOMA class
  \IfFileExists{parskip.sty}{%
    \usepackage{parskip}
  }{% else
    \setlength{\parindent}{0pt}
    \setlength{\parskip}{6pt plus 2pt minus 1pt}}
}{% if KOMA class
  \KOMAoptions{parskip=half}}
\makeatother
\usepackage{xcolor}
\IfFileExists{xurl.sty}{\usepackage{xurl}}{} % add URL line breaks if available
\IfFileExists{bookmark.sty}{\usepackage{bookmark}}{\usepackage{hyperref}}
\hypersetup{
  pdftitle={EDA},
  pdfauthor={Ignacio Vellido},
  hidelinks,
  pdfcreator={LaTeX via pandoc}}
\urlstyle{same} % disable monospaced font for URLs
\usepackage[margin=1in]{geometry}
\usepackage{color}
\usepackage{fancyvrb}
\newcommand{\VerbBar}{|}
\newcommand{\VERB}{\Verb[commandchars=\\\{\}]}
\DefineVerbatimEnvironment{Highlighting}{Verbatim}{commandchars=\\\{\}}
% Add ',fontsize=\small' for more characters per line
\usepackage{framed}
\definecolor{shadecolor}{RGB}{248,248,248}
\newenvironment{Shaded}{\begin{snugshade}}{\end{snugshade}}
\newcommand{\AlertTok}[1]{\textcolor[rgb]{0.94,0.16,0.16}{#1}}
\newcommand{\AnnotationTok}[1]{\textcolor[rgb]{0.56,0.35,0.01}{\textbf{\textit{#1}}}}
\newcommand{\AttributeTok}[1]{\textcolor[rgb]{0.77,0.63,0.00}{#1}}
\newcommand{\BaseNTok}[1]{\textcolor[rgb]{0.00,0.00,0.81}{#1}}
\newcommand{\BuiltInTok}[1]{#1}
\newcommand{\CharTok}[1]{\textcolor[rgb]{0.31,0.60,0.02}{#1}}
\newcommand{\CommentTok}[1]{\textcolor[rgb]{0.56,0.35,0.01}{\textit{#1}}}
\newcommand{\CommentVarTok}[1]{\textcolor[rgb]{0.56,0.35,0.01}{\textbf{\textit{#1}}}}
\newcommand{\ConstantTok}[1]{\textcolor[rgb]{0.00,0.00,0.00}{#1}}
\newcommand{\ControlFlowTok}[1]{\textcolor[rgb]{0.13,0.29,0.53}{\textbf{#1}}}
\newcommand{\DataTypeTok}[1]{\textcolor[rgb]{0.13,0.29,0.53}{#1}}
\newcommand{\DecValTok}[1]{\textcolor[rgb]{0.00,0.00,0.81}{#1}}
\newcommand{\DocumentationTok}[1]{\textcolor[rgb]{0.56,0.35,0.01}{\textbf{\textit{#1}}}}
\newcommand{\ErrorTok}[1]{\textcolor[rgb]{0.64,0.00,0.00}{\textbf{#1}}}
\newcommand{\ExtensionTok}[1]{#1}
\newcommand{\FloatTok}[1]{\textcolor[rgb]{0.00,0.00,0.81}{#1}}
\newcommand{\FunctionTok}[1]{\textcolor[rgb]{0.00,0.00,0.00}{#1}}
\newcommand{\ImportTok}[1]{#1}
\newcommand{\InformationTok}[1]{\textcolor[rgb]{0.56,0.35,0.01}{\textbf{\textit{#1}}}}
\newcommand{\KeywordTok}[1]{\textcolor[rgb]{0.13,0.29,0.53}{\textbf{#1}}}
\newcommand{\NormalTok}[1]{#1}
\newcommand{\OperatorTok}[1]{\textcolor[rgb]{0.81,0.36,0.00}{\textbf{#1}}}
\newcommand{\OtherTok}[1]{\textcolor[rgb]{0.56,0.35,0.01}{#1}}
\newcommand{\PreprocessorTok}[1]{\textcolor[rgb]{0.56,0.35,0.01}{\textit{#1}}}
\newcommand{\RegionMarkerTok}[1]{#1}
\newcommand{\SpecialCharTok}[1]{\textcolor[rgb]{0.00,0.00,0.00}{#1}}
\newcommand{\SpecialStringTok}[1]{\textcolor[rgb]{0.31,0.60,0.02}{#1}}
\newcommand{\StringTok}[1]{\textcolor[rgb]{0.31,0.60,0.02}{#1}}
\newcommand{\VariableTok}[1]{\textcolor[rgb]{0.00,0.00,0.00}{#1}}
\newcommand{\VerbatimStringTok}[1]{\textcolor[rgb]{0.31,0.60,0.02}{#1}}
\newcommand{\WarningTok}[1]{\textcolor[rgb]{0.56,0.35,0.01}{\textbf{\textit{#1}}}}
\usepackage{graphicx,grffile}
\makeatletter
\def\maxwidth{\ifdim\Gin@nat@width>\linewidth\linewidth\else\Gin@nat@width\fi}
\def\maxheight{\ifdim\Gin@nat@height>\textheight\textheight\else\Gin@nat@height\fi}
\makeatother
% Scale images if necessary, so that they will not overflow the page
% margins by default, and it is still possible to overwrite the defaults
% using explicit options in \includegraphics[width, height, ...]{}
\setkeys{Gin}{width=\maxwidth,height=\maxheight,keepaspectratio}
% Set default figure placement to htbp
\makeatletter
\def\fps@figure{htbp}
\makeatother
\setlength{\emergencystretch}{3em} % prevent overfull lines
\providecommand{\tightlist}{%
  \setlength{\itemsep}{0pt}\setlength{\parskip}{0pt}}
\setcounter{secnumdepth}{-\maxdimen} % remove section numbering
\usepackage{booktabs}
\usepackage{longtable}
\usepackage{array}
\usepackage{multirow}
\usepackage{wrapfig}
\usepackage{float}
\usepackage{colortbl}
\usepackage{pdflscape}
\usepackage{tabu}
\usepackage{threeparttable}
\usepackage{threeparttablex}
\usepackage[normalem]{ulem}
\usepackage{makecell}
\usepackage{xcolor}

\title{EDA}
\author{Ignacio Vellido}
\date{11/13/2020}

\begin{document}
\maketitle

\hypertarget{intro}{%
\section{Intro}\label{intro}}

Para este trabajo contamos con dos datasets distintos: \textbf{autoMPG6}
para aplicar Regresión y \textbf{haberman} para aplicar Clasificación.

\hypertarget{descripciones-de-los-problemas}{%
\subsection{Descripciones de los
problemas}\label{descripciones-de-los-problemas}}

\hypertarget{autompg6}{%
\subsubsection{autoMPG6}\label{autompg6}}

\url{http://lib.stat.cmu.edu/datasets/cars.desc}

Este dataset codifica el consumo de gasolina de distintos coches (en
millas por galón, Mpg) en base a las siguientes características:

\begin{enumerate}
\def\labelenumi{\arabic{enumi}.}
\tightlist
\item
  Displacement: Indica la cilindrada del coche, la suma del volumen útil
  de los cilindros del motor, medido en pulgadas cúbicas.
\item
  Horse\_power: Mide la potencia del coche.
\item
  Weight: Peso en libras.
\item
  Acceleration: Aceleración del coche de 0 a 60 millas por hora, medido
  en segundos.
\item
  Model\_year: Indica las dos últimas cifras del año de producción.
\end{enumerate}

El objetivo es poder predecir, en base a los cinco atributos, el consumo
Mpg de un nuevo coche:

\begin{enumerate}
\def\labelenumi{\arabic{enumi}.}
\setcounter{enumi}{5}
\tightlist
\item
  Mpg: Millas-por-galón, indica la cantidad de galones (1G +-= 3,78L) de
  fuél que consume un vehículo al recorrer una milla (1m +-= 1,6km).
\end{enumerate}

El dataset contiene 392 instancias codificanto esta información.

\begin{center}\rule{0.5\linewidth}{0.5pt}\end{center}

\hypertarget{anuxe1lisis-estaduxedstico-de-datos}{%
\section{Análisis Estadístico de
Datos}\label{anuxe1lisis-estaduxedstico-de-datos}}

\hypertarget{autompg6-1}{%
\subsubsection{autoMPG6}\label{autompg6-1}}

La descripción del problema nos da alguna información adicional sobre
las variables:

\begin{enumerate}
\def\labelenumi{\arabic{enumi}.}
\tightlist
\item
  Displacement: Variable numérica continua, contamos con valores reales
  en el rango {[}68.0,455.0{]}.
\item
  Horse\_power: Variable numérica continua, contamos con valores enteros
  en el rango {[}46,230{]}.
\item
  Weight: Variable numérica continua, contamos con valores enteros en el
  rango {[}1613,5140{]}.
\item
  Acceleration: Variable numérica continua, contamos con valores reales
  en el rango {[}8.0,24.8{]}.
\item
  Model\_year: Variable numérica discreta, contamos con valores enteros
  en el rango {[}70,82{]}.
\item
  Mpg: Variable numérica continua, contamos con valores reales en el
  rango {[}9.0,46.6{]}.
\end{enumerate}

\hypertarget{hipuxf3tesis-de-partida}{%
\paragraph{Hipótesis de partida}\label{hipuxf3tesis-de-partida}}

\begin{itemize}
\tightlist
\item
  H.1: Horse\_power puede influir en Mpg: A más potencia, más consumo.
\item
  H.2: Weight debe influir en Mpg: Un coche más pesado debería consumir
  más.
\item
  H.3: Debería haber correlación entre displacement (cilindrada) con
  horse y acceleration
\item
  H.4: Horse y acceleration podrían estar relacionadas
\item
  H.5: Viendo que contamos con un rango pequeño de años, no debería
  haber un cambio significativo de prestaciones entre años
\item
  H.6: Pero debería existir una tendencia de mejora de prestaciones con
  los años, incluyendo aumento de Displacement, Horse\_power y
  Acceleration.
\item
  H.7: Model\_year podría no mostrar relación con Mpg: Pese al paso de
  los años si contamos con diferentes tipos de vehículos (todoterrenos,
  familiares, deportivos\ldots) podría haber un consumo dispar. (Si
  existiera tendencia, viendo que los años son de las últimas décadas
  del siglo XX, podría ir el consumo hacia abajo)
\item
  H.8: Esta última hipótesis se puede aplicar al resto de variables,
  indicándonos que Model\_year no debería tener relevancia para este
  problema de regresión.
\item
  H.9: Horse\_power podría depender de las variables Displacement y
  Weight
\end{itemize}

\begin{center}\rule{0.5\linewidth}{0.5pt}\end{center}

Cargamos los datos:

\begin{Shaded}
\begin{Highlighting}[]
\NormalTok{names <-}\StringTok{ }\KeywordTok{c}\NormalTok{(}\StringTok{"Displacement"}\NormalTok{, }\StringTok{"Horse_power"}\NormalTok{, }\StringTok{"Weight"}\NormalTok{, }\StringTok{"Acceleration"}\NormalTok{, }\StringTok{"Model_year"}\NormalTok{, }\StringTok{"Mpg"}\NormalTok{)}

\NormalTok{auto <-}\StringTok{ }\KeywordTok{read_csv}\NormalTok{(}\StringTok{"Data/autoMPG6/autoMPG6.dat"}\NormalTok{, }\DataTypeTok{comment =} \StringTok{"@"}\NormalTok{, }\DataTypeTok{col_names =}\NormalTok{ names)}
\end{Highlighting}
\end{Shaded}

\begin{verbatim}

-- Column specification --------------------------------------------------------
cols(
  Displacement = col_double(),
  Horse_power = col_double(),
  Weight = col_double(),
  Acceleration = col_double(),
  Model_year = col_double(),
  Mpg = col_double()
)
\end{verbatim}

Antes de comenzar a analizar las variables nos plantemos una cuestión:
¿Debemos considerar Model\_year como una variable numérica o como un
factor categórico? Aunque por la hipótesis H.7 podríamos acabar no
eligiendo la variable para el problema, es necesario plantearse esta
cuestión antes de comenzar.

Sabemos que las observaciones para esta variable cuenta con valores
entre 72 y 82, por lo que tenemos información exacta del año (en
comparación, por ejemplo, con agrupaciones mayores como la década o el
siglo). El hecho de tratarla como categórica o quantitativa depende
mucho del problema. En este caso, tenemos interés en cuestionarnos por
valores entre años, por ejemplo, el consumo entre los años 75 y 76 (por
otro lado, no tenemos información más precisa para los meses dentro del
año)

En un principio, el dataset está planteado para regresión, por lo que
tendríamos dos opciones: - Mantenerlo como categórico y generar
variables dummy (Valores 0-1 para indicar si la instancia es de ese
año). Suponiendo que tenemos al menos una instancia de cada año, esto
nos generaría 12 variables nuevas. - Mantenerlo como numérico, pero
teniendo cuidado de cómo interpretar el año.

Proseguimos con tanto dejando Model\_year como variable numérica.

\hypertarget{anuxe1lisis-univariable}{%
\paragraph{Análisis univariable}\label{anuxe1lisis-univariable}}

\begin{Shaded}
\begin{Highlighting}[]
\KeywordTok{head}\NormalTok{(auto)}
\end{Highlighting}
\end{Shaded}

\begin{tabular}{r|r|r|r|r|r}
\hline
Displacement & Horse\_power & Weight & Acceleration & Model\_year & Mpg\\
\hline
91 & 70 & 1955 & 20.5 & 71 & 26.0\\
\hline
232 & 100 & 2789 & 15.0 & 73 & 18.0\\
\hline
350 & 145 & 4055 & 12.0 & 76 & 13.0\\
\hline
318 & 140 & 4080 & 13.7 & 78 & 17.5\\
\hline
113 & 95 & 2372 & 15.0 & 70 & 24.0\\
\hline
97 & 60 & 1834 & 19.0 & 71 & 27.0\\
\hline
\end{tabular}

Hacemos summary para sacar datos de relevancia

\begin{Shaded}
\begin{Highlighting}[]
\KeywordTok{summary}\NormalTok{(auto)}
\end{Highlighting}
\end{Shaded}

\begin{verbatim}
  Displacement    Horse_power        Weight      Acceleration     Model_year   
 Min.   : 68.0   Min.   : 46.0   Min.   :1613   Min.   : 8.00   Min.   :70.00  
 1st Qu.:105.0   1st Qu.: 75.0   1st Qu.:2225   1st Qu.:13.78   1st Qu.:73.00  
 Median :151.0   Median : 93.5   Median :2804   Median :15.50   Median :76.00  
 Mean   :194.4   Mean   :104.5   Mean   :2978   Mean   :15.54   Mean   :75.98  
 3rd Qu.:275.8   3rd Qu.:126.0   3rd Qu.:3615   3rd Qu.:17.02   3rd Qu.:79.00  
 Max.   :455.0   Max.   :230.0   Max.   :5140   Max.   :24.80   Max.   :82.00  
      Mpg       
 Min.   : 9.00  
 1st Qu.:17.00  
 Median :22.75  
 Mean   :23.45  
 3rd Qu.:29.00  
 Max.   :46.60  
\end{verbatim}

El dataset no cuenta con valores repetidos

\begin{Shaded}
\begin{Highlighting}[]
\KeywordTok{sum}\NormalTok{(}\KeywordTok{duplicated}\NormalTok{(auto))}
\end{Highlighting}
\end{Shaded}

\begin{verbatim}
[1] 0
\end{verbatim}

Ni missing values

\begin{Shaded}
\begin{Highlighting}[]
\KeywordTok{sum}\NormalTok{(}\KeywordTok{is.na}\NormalTok{(auto))}
\end{Highlighting}
\end{Shaded}

\begin{verbatim}
[1] 0
\end{verbatim}

Vamos a sacar plots de cada variable para verlo mejor

\begin{Shaded}
\begin{Highlighting}[]
\KeywordTok{ggplot}\NormalTok{(}\KeywordTok{gather}\NormalTok{(auto), }\KeywordTok{aes}\NormalTok{(value)) }\OperatorTok{+}
\StringTok{  }\KeywordTok{geom_histogram}\NormalTok{(}\DataTypeTok{bins =} \DecValTok{15}\NormalTok{, }\DataTypeTok{color=}\StringTok{"white"}\NormalTok{) }\OperatorTok{+}
\StringTok{  }\KeywordTok{facet_wrap}\NormalTok{(}\OperatorTok{~}\NormalTok{key, }\DataTypeTok{scales =} \StringTok{'free_x'}\NormalTok{) }\OperatorTok{+}
\StringTok{  }\KeywordTok{theme_light}\NormalTok{() }\OperatorTok{+}
\StringTok{  }\KeywordTok{theme}\NormalTok{(}\DataTypeTok{strip.background =} \KeywordTok{element_rect}\NormalTok{(}\DataTypeTok{fill=}\StringTok{"grey"}\NormalTok{, }\DataTypeTok{size=}\DecValTok{2}\NormalTok{))}\OperatorTok{+}
\StringTok{  }\KeywordTok{theme}\NormalTok{(}\DataTypeTok{strip.text =} \KeywordTok{element_text}\NormalTok{(}\DataTypeTok{colour =} \StringTok{'black'}\NormalTok{)) }\OperatorTok{+}
\StringTok{  }\KeywordTok{labs}\NormalTok{(}\DataTypeTok{title=}\StringTok{"Histogramas de cada variable"}\NormalTok{, }\DataTypeTok{x =} \StringTok{""}\NormalTok{)}
\end{Highlighting}
\end{Shaded}

\begin{center}\includegraphics{EDA_files/figure-latex/unnamed-chunk-6-1} \end{center}

Una a una

\begin{Shaded}
\begin{Highlighting}[]
\NormalTok{colors <-}\StringTok{ }\KeywordTok{c}\NormalTok{(}\StringTok{"chocolate"}\NormalTok{, }\StringTok{"deepskyblue1"}\NormalTok{, }\StringTok{"plum1"}\NormalTok{, }\StringTok{"hotpink4"}\NormalTok{, }\StringTok{"orange"}\NormalTok{, }\StringTok{"springgreen4"}\NormalTok{)}
\NormalTok{bins <-}\StringTok{ }\KeywordTok{c}\NormalTok{(}\DecValTok{10}\NormalTok{,}\DecValTok{10}\NormalTok{,}\DecValTok{15}\NormalTok{,}\DecValTok{15}\NormalTok{,}\DecValTok{14}\NormalTok{,}\DecValTok{18}\NormalTok{)}
\NormalTok{plt <-}\StringTok{ }\KeywordTok{list}\NormalTok{(}\DataTypeTok{length =} \KeywordTok{length}\NormalTok{(names))}

\ControlFlowTok{for}\NormalTok{ (i }\ControlFlowTok{in} \DecValTok{1}\OperatorTok{:}\KeywordTok{length}\NormalTok{(names)) \{}
  \KeywordTok{ggplot}\NormalTok{(auto, }\KeywordTok{aes_string}\NormalTok{(}\DataTypeTok{x=}\NormalTok{names[i])) }\OperatorTok{+}\StringTok{ }
\StringTok{    }\KeywordTok{geom_histogram}\NormalTok{(}\KeywordTok{aes}\NormalTok{(}\DataTypeTok{y=}\NormalTok{..density..), }\DataTypeTok{bins=}\NormalTok{bins[i], }\DataTypeTok{color=}\StringTok{"white"}\NormalTok{, }\DataTypeTok{fill=}\NormalTok{colors[i]) }\OperatorTok{+}
\StringTok{    }\KeywordTok{geom_density}\NormalTok{(}\DataTypeTok{alpha=}\NormalTok{.}\DecValTok{3}\NormalTok{, }\DataTypeTok{fill=}\StringTok{"black"}\NormalTok{, }\DataTypeTok{size=}\DecValTok{1}\NormalTok{) }\OperatorTok{+}
\StringTok{    }\KeywordTok{labs}\NormalTok{(}\DataTypeTok{title=}\StringTok{""}\NormalTok{, }\DataTypeTok{x=}\StringTok{""}\NormalTok{, }\DataTypeTok{y=}\StringTok{""}\NormalTok{) }\OperatorTok{+}
\StringTok{    }\KeywordTok{theme_light}\NormalTok{() ->}\StringTok{ }\NormalTok{plt[[i]]}
  
  \KeywordTok{print}\NormalTok{(plt[[i]] }\OperatorTok{+}\StringTok{ }\KeywordTok{labs}\NormalTok{(}\DataTypeTok{title=}\KeywordTok{sprintf}\NormalTok{(}\StringTok{"Histograma %s"}\NormalTok{, names[i]), }\DataTypeTok{x=}\StringTok{""}\NormalTok{))}
\NormalTok{\}}
\end{Highlighting}
\end{Shaded}

\begin{center}\includegraphics{EDA_files/figure-latex/unnamed-chunk-7-1} \end{center}

\begin{center}\includegraphics{EDA_files/figure-latex/unnamed-chunk-7-2} \end{center}

\begin{center}\includegraphics{EDA_files/figure-latex/unnamed-chunk-7-3} \end{center}

\begin{center}\includegraphics{EDA_files/figure-latex/unnamed-chunk-7-4} \end{center}

\begin{center}\includegraphics{EDA_files/figure-latex/unnamed-chunk-7-5} \end{center}

\begin{center}\includegraphics{EDA_files/figure-latex/unnamed-chunk-7-6} \end{center}

\begin{Shaded}
\begin{Highlighting}[]
\KeywordTok{plot_grid}\NormalTok{(}\DataTypeTok{plotlist=}\NormalTok{plt, }\DataTypeTok{ncol=}\DecValTok{2}\NormalTok{, }\DataTypeTok{labels =}\NormalTok{ names, }\DataTypeTok{label_size =} \DecValTok{8}\NormalTok{)}
\end{Highlighting}
\end{Shaded}

\begin{center}\includegraphics{EDA_files/figure-latex/unnamed-chunk-7-7} \end{center}

\begin{Shaded}
\begin{Highlighting}[]
\NormalTok{colors <-}\StringTok{ }\KeywordTok{c}\NormalTok{(}\StringTok{"chocolate"}\NormalTok{, }\StringTok{"deepskyblue1"}\NormalTok{, }\StringTok{"plum1"}\NormalTok{, }\StringTok{"hotpink4"}\NormalTok{, }\StringTok{"orange"}\NormalTok{, }\StringTok{"springgreen4"}\NormalTok{)}
\NormalTok{plt <-}\StringTok{ }\KeywordTok{list}\NormalTok{(}\DataTypeTok{length =} \KeywordTok{length}\NormalTok{(names))}

\ControlFlowTok{for}\NormalTok{ (i }\ControlFlowTok{in} \DecValTok{1}\OperatorTok{:}\KeywordTok{length}\NormalTok{(names)) \{}
  \KeywordTok{ggplot}\NormalTok{(auto, }\KeywordTok{aes_string}\NormalTok{(}\DataTypeTok{x=}\NormalTok{names[i])) }\OperatorTok{+}\StringTok{ }
\StringTok{    }\KeywordTok{geom_boxplot}\NormalTok{(}\DataTypeTok{fill =}\NormalTok{ colors[i]) }\OperatorTok{+}
\StringTok{    }\KeywordTok{labs}\NormalTok{(}\DataTypeTok{title=}\StringTok{""}\NormalTok{, }\DataTypeTok{x=}\StringTok{""}\NormalTok{, }\DataTypeTok{y=}\StringTok{""}\NormalTok{) }\OperatorTok{+}
\StringTok{    }\KeywordTok{theme_light}\NormalTok{() ->}\StringTok{ }\NormalTok{plt[[i]]}
  
  \KeywordTok{print}\NormalTok{(plt[[i]] }\OperatorTok{+}\StringTok{ }\KeywordTok{labs}\NormalTok{(}\DataTypeTok{title=}\KeywordTok{sprintf}\NormalTok{(}\StringTok{"Boxplot %s"}\NormalTok{, names[i]), }\DataTypeTok{x=}\StringTok{""}\NormalTok{))}
\NormalTok{\}}
\end{Highlighting}
\end{Shaded}

\begin{center}\includegraphics{EDA_files/figure-latex/unnamed-chunk-8-1} \end{center}

\begin{center}\includegraphics{EDA_files/figure-latex/unnamed-chunk-8-2} \end{center}

\begin{center}\includegraphics{EDA_files/figure-latex/unnamed-chunk-8-3} \end{center}

\begin{center}\includegraphics{EDA_files/figure-latex/unnamed-chunk-8-4} \end{center}

\begin{center}\includegraphics{EDA_files/figure-latex/unnamed-chunk-8-5} \end{center}

\begin{center}\includegraphics{EDA_files/figure-latex/unnamed-chunk-8-6} \end{center}

\begin{Shaded}
\begin{Highlighting}[]
\KeywordTok{plot_grid}\NormalTok{(}\DataTypeTok{plotlist=}\NormalTok{plt, }\DataTypeTok{ncol=}\DecValTok{2}\NormalTok{, }\DataTypeTok{labels =}\NormalTok{ names, }\DataTypeTok{label_size =} \DecValTok{8}\NormalTok{)}
\end{Highlighting}
\end{Shaded}

\begin{center}\includegraphics{EDA_files/figure-latex/unnamed-chunk-8-7} \end{center}

\begin{Shaded}
\begin{Highlighting}[]
\KeywordTok{ggplot}\NormalTok{(}\KeywordTok{melt}\NormalTok{(auto), }\KeywordTok{aes}\NormalTok{(}\DataTypeTok{x=}\NormalTok{variable, }\DataTypeTok{y=}\NormalTok{value)) }\OperatorTok{+}\StringTok{ }
\StringTok{  }\KeywordTok{geom_boxplot}\NormalTok{() }\OperatorTok{+}
\StringTok{  }\KeywordTok{labs}\NormalTok{(}\DataTypeTok{title=}\StringTok{"Boxplot con mismo rango"}\NormalTok{) }\OperatorTok{+}
\StringTok{  }\KeywordTok{theme}\NormalTok{(}\DataTypeTok{axis.text.x =} \KeywordTok{element_text}\NormalTok{(}\DataTypeTok{angle =} \DecValTok{90}\NormalTok{))}
\end{Highlighting}
\end{Shaded}

\begin{verbatim}
No id variables; using all as measure variables
\end{verbatim}

\begin{center}\includegraphics{EDA_files/figure-latex/unnamed-chunk-9-1} \end{center}

\begin{Shaded}
\begin{Highlighting}[]
\NormalTok{auto }\OperatorTok\StringTok{ }\NormalTok{dplyr}\OperatorTok{::}\KeywordTok{select}\NormalTok{(}\OperatorTok{-}\NormalTok{Weight) }\OperatorTok\StringTok{ }
\StringTok{  }\KeywordTok{melt}\NormalTok{() }\OperatorTok\StringTok{ }
\StringTok{  }\KeywordTok{ggplot}\NormalTok{(}\KeywordTok{aes}\NormalTok{(}\DataTypeTok{x=}\NormalTok{variable, }\DataTypeTok{y=}\NormalTok{value)) }\OperatorTok{+}\StringTok{ }
\StringTok{    }\KeywordTok{geom_boxplot}\NormalTok{() }\OperatorTok{+}
\StringTok{    }\KeywordTok{labs}\NormalTok{(}\DataTypeTok{title=}\StringTok{"Boxplot sin Weight"}\NormalTok{) }\OperatorTok{+}
\StringTok{    }\KeywordTok{theme}\NormalTok{(}\DataTypeTok{axis.text.x =} \KeywordTok{element_text}\NormalTok{(}\DataTypeTok{angle =} \DecValTok{90}\NormalTok{))}
\end{Highlighting}
\end{Shaded}

\begin{verbatim}
No id variables; using all as measure variables
\end{verbatim}

\begin{center}\includegraphics{EDA_files/figure-latex/unnamed-chunk-9-2} \end{center}

Ya la descripción del problema nos lo decía, los rangos en los que se
distribuyen los datos son muy diferentes dependiendo de la variable. Se
pueden estandarizar los datos para solucionar este problema, aunque para
regresión lineal no es necesario (sí lo es para KNN)

Podemos comparar los rangos intercuartiles si estandarizamos antes el
dataset

\begin{Shaded}
\begin{Highlighting}[]
\KeywordTok{scale}\NormalTok{(auto) }\OperatorTok\StringTok{ }\KeywordTok{apply}\NormalTok{(}\DecValTok{2}\NormalTok{, IQR)}
\end{Highlighting}
\end{Shaded}

\begin{verbatim}
Displacement  Horse_power       Weight Acceleration   Model_year          Mpg 
    1.631723     1.324980     1.635856     1.178021     1.628781     1.537475 
\end{verbatim}

También podemos ver la distancia entre mínimos y máximos

\begin{Shaded}
\begin{Highlighting}[]
\KeywordTok{scale}\NormalTok{(auto) }\OperatorTok\StringTok{ }\KeywordTok{apply}\NormalTok{(}\DecValTok{2}\NormalTok{, range) }\OperatorTok\StringTok{ }\KeywordTok{apply}\NormalTok{(}\DecValTok{2}\NormalTok{, dist)}
\end{Highlighting}
\end{Shaded}

\begin{verbatim}
Displacement  Horse_power       Weight Acceleration   Model_year          Mpg 
    3.698253     4.780318     4.152330     6.089463     3.257562     4.817420 
\end{verbatim}

\hypertarget{displacement}{%
\subparagraph{Displacement:}\label{displacement}}

(coger los plots de la variable sola)

La cilindrada vemos con una desviación grande y una gran concentración
en los valores inferiores. Desviado a la izquierda, no parece seguir una
distribución normal. Existe una alta concentración en torno al valor
125, muy por encima del recuento que alcanzan el resto de valores

\hypertarget{horse_power}{%
\paragraph{Horse\_power}\label{horse_power}}

Similar a Displacement pero cuenta con una mayor dispersión y algunos
valores muya altos. A día de hoy los coches suelen rondar los 120 en
turismos y los 200 en SUVs. Aquí contamos con predominancia en el rango
aproximado {[}70, 125{]} con algunas instancias por encima de los 200.
Desviado a la izquierda, no parece seguir una distribución normal.

\hypertarget{weight}{%
\paragraph{Weight}\label{weight}}

Una distribución más achatada que las anteriores, también ladeada hacia
la izquierda. Un rango mayor

\hypertarget{acceleration}{%
\paragraph{Acceleration}\label{acceleration}}

Valores altamentes concentrados pero en general con un rango alto.
Parece seguir una distribución normal.

\hypertarget{model_year}{%
\paragraph{Model\_year}\label{model_year}}

Aunque no se vea bien en las gráficas, contamos con valores de todos los
años, más o menos equitativamente

\begin{Shaded}
\begin{Highlighting}[]
\KeywordTok{table}\NormalTok{(auto}\OperatorTok{$}\NormalTok{Model_year)}
\end{Highlighting}
\end{Shaded}

\begin{verbatim}

70 71 72 73 74 75 76 77 78 79 80 81 82 
29 27 28 40 26 30 34 28 36 29 27 28 30 
\end{verbatim}

\begin{center}\rule{0.5\linewidth}{0.5pt}\end{center}

\hypertarget{anuxe1lisis-sobre-las-distribuciones}{%
\subsubsection{Análisis sobre las
distribuciones}\label{anuxe1lisis-sobre-las-distribuciones}}

Hemos comentado antes que no apreciamos semejanzas con una distribución
normal en algunas de las variables, lo comprobamos con un test
estadístico (Shapiro-Wilk test):

\begin{Shaded}
\begin{Highlighting}[]
\KeywordTok{normality}\NormalTok{(auto) }\OperatorTok\StringTok{ }\KeywordTok{filter}\NormalTok{(p_value }\OperatorTok{<}\StringTok{ }\FloatTok{0.05}\NormalTok{)}
\end{Highlighting}
\end{Shaded}

\begin{verbatim}
Warning: `cols` is now required when using unnest().
Please use `cols = c(statistic)`
\end{verbatim}

\begin{tabular}{l|r|r|r}
\hline
vars & statistic & p\_value & sample\\
\hline
Displacement & 0.8818359 & 0.0000000 & 392\\
\hline
Horse\_power & 0.9040975 & 0.0000000 & 392\\
\hline
Weight & 0.9414661 & 0.0000000 & 392\\
\hline
Acceleration & 0.9918671 & 0.0305289 & 392\\
\hline
Model\_year & 0.9469666 & 0.0000000 & 392\\
\hline
Mpg & 0.9671696 & 0.0000001 & 392\\
\hline
\end{tabular}

El test de Shapiro nos dice que ninguna variable sigue una distribución
normal, con bastante certeza excepto en Acceleration.

Para regresión aún así no es necesario.

Se muestra aquí como no hay que dejarse engañar por los gráficos, puesto
que Acceleration parecía seguirla. El p-value de Acceleration está muy
cerca del umbral (0.03 vs 0.05). Es bastante probable de que la parte
central derecha de la distribución sea la causante de no asegurar la
normalidad.

Vamos a mostrarlo con gráficos Q-Q para verlo mejor:

\begin{Shaded}
\begin{Highlighting}[]
\NormalTok{plt <-}\StringTok{ }\KeywordTok{list}\NormalTok{(}\DataTypeTok{length =} \KeywordTok{length}\NormalTok{(names))}

\NormalTok{x<-}\KeywordTok{rnorm}\NormalTok{(}\DecValTok{100}\NormalTok{, }\DataTypeTok{mean=}\DecValTok{0}\NormalTok{, }\DataTypeTok{sd=}\DecValTok{1}\NormalTok{)}

\ControlFlowTok{for}\NormalTok{ (i }\ControlFlowTok{in} \DecValTok{1}\OperatorTok{:}\KeywordTok{length}\NormalTok{(names)) \{}
  \KeywordTok{ggplot}\NormalTok{(auto, }\KeywordTok{aes_string}\NormalTok{(}\DataTypeTok{sample=}\NormalTok{names[i])) }\OperatorTok{+}\StringTok{ }
\StringTok{    }\KeywordTok{stat_qq}\NormalTok{(}\DataTypeTok{alpha=}\NormalTok{.}\DecValTok{3}\NormalTok{, }\DataTypeTok{fill=}\NormalTok{colors[i], }\DataTypeTok{size=}\DecValTok{1}\NormalTok{) }\OperatorTok{+}
\StringTok{    }\KeywordTok{stat_qq_line}\NormalTok{() }\OperatorTok{+}
\StringTok{    }\KeywordTok{labs}\NormalTok{(}\DataTypeTok{title=}\StringTok{""}\NormalTok{, }\DataTypeTok{x=}\StringTok{""}\NormalTok{, }\DataTypeTok{y=}\StringTok{""}\NormalTok{) }\OperatorTok{+}
\StringTok{    }\KeywordTok{theme_light}\NormalTok{() ->}\StringTok{ }\NormalTok{plt[[i]]}
  
  \KeywordTok{print}\NormalTok{(plt[[i]] }\OperatorTok{+}\StringTok{ }\KeywordTok{labs}\NormalTok{(}\DataTypeTok{title=}\KeywordTok{sprintf}\NormalTok{(}\StringTok{"QQ-plot %s"}\NormalTok{, names[i]), }\DataTypeTok{x=}\StringTok{""}\NormalTok{))}
\NormalTok{\}}
\end{Highlighting}
\end{Shaded}

\begin{center}\includegraphics{EDA_files/figure-latex/unnamed-chunk-14-1} \end{center}

\begin{center}\includegraphics{EDA_files/figure-latex/unnamed-chunk-14-2} \end{center}

\begin{center}\includegraphics{EDA_files/figure-latex/unnamed-chunk-14-3} \end{center}

\begin{center}\includegraphics{EDA_files/figure-latex/unnamed-chunk-14-4} \end{center}

\begin{center}\includegraphics{EDA_files/figure-latex/unnamed-chunk-14-5} \end{center}

\begin{center}\includegraphics{EDA_files/figure-latex/unnamed-chunk-14-6} \end{center}

\begin{Shaded}
\begin{Highlighting}[]
\KeywordTok{plot_grid}\NormalTok{(}\DataTypeTok{plotlist=}\NormalTok{plt, }\DataTypeTok{ncol=}\DecValTok{2}\NormalTok{, }\DataTypeTok{labels =}\NormalTok{ names, }\DataTypeTok{label_size =} \DecValTok{8}\NormalTok{)}
\end{Highlighting}
\end{Shaded}

\begin{center}\includegraphics{EDA_files/figure-latex/unnamed-chunk-14-7} \end{center}

Estos gráficos Q-Q nos muestran más claramente que las variables no
siguen distribuciones normales. La distribución de Acceleration es la
que más se asemeja y eso lo vemos en el estadístico de Shapiro, pero en
la cola superior existe una diferencia significativa que hace que el
test rechace.

Skewness:

\begin{Shaded}
\begin{Highlighting}[]
\NormalTok{skewCols <-}\StringTok{ }\KeywordTok{find_skewness}\NormalTok{(auto)}
\KeywordTok{colnames}\NormalTok{(auto)[skewCols]}
\end{Highlighting}
\end{Shaded}

\begin{verbatim}
[1] "Displacement" "Horse_power"  "Weight"      
\end{verbatim}

\begin{Shaded}
\begin{Highlighting}[]
\KeywordTok{cat}\NormalTok{(}\StringTok{"Displacement: "}\NormalTok{)}
\KeywordTok{skewness}\NormalTok{(auto}\OperatorTok{$}\NormalTok{Displacement)}
\KeywordTok{cat}\NormalTok{(}\StringTok{"Horse_power: "}\NormalTok{)}
\KeywordTok{skewness}\NormalTok{(auto}\OperatorTok{$}\NormalTok{Horse_power)}
\KeywordTok{cat}\NormalTok{(}\StringTok{"Weight: "}\NormalTok{)}
\KeywordTok{skewness}\NormalTok{(auto}\OperatorTok{$}\NormalTok{Weight)}
\KeywordTok{cat}\NormalTok{(}\StringTok{"Mpg: "}\NormalTok{)}
\KeywordTok{skewness}\NormalTok{(auto}\OperatorTok{$}\NormalTok{Mpg)}
\end{Highlighting}
\end{Shaded}

\begin{verbatim}
Displacement: [1] 0.6989813
Horse_power: [1] 1.083161
Weight: [1] 0.5175953
Mpg: [1] 0.4553414
\end{verbatim}

Sobre la skewness, tal y como se había visto en las gráficas, algunas de
las variables la tienen, en los 3 casos positivas (hacia la izquierda).

Los plots nos han dado idea de que Mpg tiene cierta skewness, pero cae
por debajo del umbral de 0.5.

\begin{center}\rule{0.5\linewidth}{0.5pt}\end{center}

\hypertarget{transformaciones}{%
\subsubsection{Transformaciones}\label{transformaciones}}

Tampoco vemos necesario crear variables nuevas a partir de las vistas,
por el conocimiento que tenemos del problema parece que las variables
son coherentes.

Las transformaciones necesarias para pasar a una distribución normal
dependen de la variable en cuestion. Primero debemos averiguar que tipo
de distribución siguen.

De todas maneras, los métodos utilizados para regresión (regresión
lineal y KNN) no asumen ninguna forma para la distribución de los datos,
por lo que no es necesario aplicar nada.

Algunas parecen tener una distribución exponencial

\begin{Shaded}
\begin{Highlighting}[]
\CommentTok{# auto_transform <- preProcess(auto[,skewCols], method=c("YeoJohnson"))}
\CommentTok{# auto_norm <- predict(auto_transform, auto[,skewCols])}

\CommentTok{# auto_transform <- preProcess(auto[,1:6], method=c("scale", "center"))}
\NormalTok{auto_transform <-}\StringTok{ }\KeywordTok{preProcess}\NormalTok{(auto[,}\DecValTok{1}\OperatorTok{:}\DecValTok{6}\NormalTok{], }\DataTypeTok{method=}\KeywordTok{c}\NormalTok{(}\StringTok{"YeoJohnson"}\NormalTok{,}\StringTok{"scale"}\NormalTok{, }\StringTok{"center"}\NormalTok{))}
\NormalTok{auto_norm <-}\StringTok{ }\KeywordTok{predict}\NormalTok{(auto_transform, auto[,}\DecValTok{1}\OperatorTok{:}\DecValTok{6}\NormalTok{])}

\KeywordTok{summary}\NormalTok{(auto_norm)}
\end{Highlighting}
\end{Shaded}

\begin{verbatim}
  Displacement      Horse_power           Weight          Acceleration     
 Min.   :-1.8856   Min.   :-2.59657   Min.   :-2.19582   Min.   :-3.08413  
 1st Qu.:-0.8847   1st Qu.:-0.77280   1st Qu.:-0.88990   1st Qu.:-0.61672  
 Median :-0.1367   Median :-0.07186   Median :-0.02988   Median : 0.02508  
 Mean   : 0.0000   Mean   : 0.00000   Mean   : 0.00000   Mean   : 0.00000  
 3rd Qu.: 0.9411   3rd Qu.: 0.77159   3rd Qu.: 0.84752   3rd Qu.: 0.56609  
 Max.   : 1.7103   Max.   : 2.16086   Max.   : 1.95352   Max.   : 3.03480  
   Model_year           Mpg          
 Min.   :-1.6431   Min.   :-2.45780  
 1st Qu.:-0.8047   1st Qu.:-0.79553  
 Median : 0.0172   Median : 0.04412  
 Mean   : 0.0000   Mean   : 0.00000  
 3rd Qu.: 0.8236   3rd Qu.: 0.78143  
 Max.   : 1.6154   Max.   : 2.32155  
\end{verbatim}

Para la variable Acceleration aplicando una transformación de YeoJohnson
es suficiente para pasarla a una normal.

\begin{Shaded}
\begin{Highlighting}[]
\KeywordTok{normality}\NormalTok{(auto)}
\end{Highlighting}
\end{Shaded}

\begin{verbatim}
Warning: `cols` is now required when using unnest().
Please use `cols = c(statistic)`
\end{verbatim}

\begin{tabular}{l|r|r|r}
\hline
vars & statistic & p\_value & sample\\
\hline
Displacement & 0.8818359 & 0.0000000 & 392\\
\hline
Horse\_power & 0.9040975 & 0.0000000 & 392\\
\hline
Weight & 0.9414661 & 0.0000000 & 392\\
\hline
Acceleration & 0.9918671 & 0.0305289 & 392\\
\hline
Model\_year & 0.9469666 & 0.0000000 & 392\\
\hline
Mpg & 0.9671696 & 0.0000001 & 392\\
\hline
\end{tabular}

Aunque para regresión lineal no es absolutamente necesario, podemos
estandarizar los datos a media 0 y dev 1, facilitando un poco los
cálculos. La inferencia estadística de la regresión no va a variar, por
lo que es conveniente hacerlo. Haciendo esto debemos tener cuidado a la
hora de interpretar los resultados de la regresión para no confundirnos.

\begin{center}\rule{0.5\linewidth}{0.5pt}\end{center}

\hypertarget{outliers}{%
\subsubsection{Outliers}\label{outliers}}

Como hemos visto anteriormente en los boxplots, las únicas variables con
valores muy alejados del centro de la distribución son Acceleration y
Horse\_power.

Por el significado del problema, probablemente estos posibles outliers
correspondan a coches de alta gama o potentes en la época. Esto tampoco
lo podemos asegurar puesto que contamos con pocas características, pero
se considera un razonamiento coherente. Además, puesto que los valores
caen dentro de los rangos posibles para coches de la época, podemos
descartar que sean errores de medida.

Deberíamos decidir si mantener o no estas instancias. Como en nuestro
caso se nos ha pedido predecir el consumo Mpg, sin darnos
consideraciones sobre los tipos/gamas de coches a los que se enfoca,
proseguimos dejándo estas filas.

\begin{center}\rule{0.5\linewidth}{0.5pt}\end{center}

\hypertarget{anuxe1lisis-de-correlaciuxf3n}{%
\subsubsection{Análisis de
correlación}\label{anuxe1lisis-de-correlaciuxf3n}}

Tenemos que tener en cuenta que las variables no siguen distribuciones
normales. Aunque el coeficiente de Pearson no asume normalidad (si asume
varianza y covarianza finitas), podemos usar el coeficiente de Kendall
para los cálculos. Independientemente del método usado vamos a obtener
las mismas correlaciones en este dataset, solo varía la fuerza con la
que se dan.

Para regresión la correlación en los datos no es preocupante. Al
contrario, podría haber información (poca, pero alguna cantidad) que se
aporte y nos ayude en el problema. Además, la propia metodología de
selección de variables en el modelo multivariable nos ayudará a
descartar aquellas variables que no sean necesarias como regresor.

Corrplot

\begin{Shaded}
\begin{Highlighting}[]
\KeywordTok{corrplot.mixed}\NormalTok{(}\KeywordTok{cor}\NormalTok{(auto), }\DataTypeTok{tl.pos=}\StringTok{"lt"}\NormalTok{, }\DataTypeTok{upper=}\StringTok{"color"}\NormalTok{, }\DataTypeTok{title=}\StringTok{"Pearson"}\NormalTok{)}
\end{Highlighting}
\end{Shaded}

\begin{center}\includegraphics{EDA_files/figure-latex/unnamed-chunk-19-1} \end{center}

\begin{Shaded}
\begin{Highlighting}[]
\KeywordTok{corrplot.mixed}\NormalTok{(}\KeywordTok{cor}\NormalTok{(auto, }\DataTypeTok{method=}\StringTok{"kendall"}\NormalTok{), }\DataTypeTok{tl.pos=}\StringTok{"lt"}\NormalTok{, }\DataTypeTok{upper=}\StringTok{"color"}\NormalTok{, }\DataTypeTok{title=}\StringTok{"Kendall"}\NormalTok{)}
\end{Highlighting}
\end{Shaded}

\begin{center}\includegraphics{EDA_files/figure-latex/unnamed-chunk-19-2} \end{center}

Estas gráficas nos dicen que existe una alta correlación en el dataset,
generalmente entre todas las variables (a excepción de Model\_year),
pero extremadamente fuerte en las parejas:

\begin{enumerate}
\def\labelenumi{\arabic{enumi}.}
\tightlist
\item
  Horse\_power \& Displacement
\item
  Weight \& Displacement
\item
  Weight \& Horse\_power
\item
  Acceleration \& Horse\_power
\item
  Mpg \& Horse\_power
\item
  Mpg \& Displacement
\item
  Mpg \& Weight
\end{enumerate}

\begin{Shaded}
\begin{Highlighting}[]
\KeywordTok{scatterplotMatrix}\NormalTok{(auto, }\DataTypeTok{pch=}\DecValTok{20}\NormalTok{, }\DataTypeTok{col=}\StringTok{"deepskyblue"}\NormalTok{)}
\end{Highlighting}
\end{Shaded}

\begin{center}\includegraphics{EDA_files/figure-latex/unnamed-chunk-20-1} \end{center}

El scatterplot anterior nos muestra mejor la forma de estas
correlaciones. Vemos que en todos los casos en los que se da una
correlación positiva existe una tendencia lineal entre los datos de
ambas variables, y en las negativas una tendencia logarítmica.

Vamos a mostrar algunas Positivas:

\begin{Shaded}
\begin{Highlighting}[]
\KeywordTok{ggplot}\NormalTok{(auto, }\KeywordTok{aes}\NormalTok{(}\DataTypeTok{x=}\NormalTok{Horse_power, }\DataTypeTok{y=}\NormalTok{Displacement)) }\OperatorTok{+}
\StringTok{  }\KeywordTok{geom_point}\NormalTok{() }\OperatorTok{+}
\StringTok{  }\KeywordTok{geom_smooth}\NormalTok{(}\DataTypeTok{formula =}\NormalTok{ y}\OperatorTok{~}\NormalTok{x, }\DataTypeTok{method=}\NormalTok{glm) }\OperatorTok{+}
\StringTok{  }\KeywordTok{labs}\NormalTok{(}\DataTypeTok{title=}\StringTok{"Relación Horse_power-Displacement"}\NormalTok{) }\OperatorTok{+}
\StringTok{  }\KeywordTok{theme_light}\NormalTok{()}
\end{Highlighting}
\end{Shaded}

\begin{center}\includegraphics{EDA_files/figure-latex/unnamed-chunk-21-1} \end{center}

\begin{Shaded}
\begin{Highlighting}[]
\KeywordTok{ggplot}\NormalTok{(auto, }\KeywordTok{aes}\NormalTok{(}\DataTypeTok{x=}\NormalTok{Weight, }\DataTypeTok{y=}\NormalTok{Displacement)) }\OperatorTok{+}
\StringTok{  }\KeywordTok{geom_point}\NormalTok{() }\OperatorTok{+}
\StringTok{  }\KeywordTok{geom_smooth}\NormalTok{(}\DataTypeTok{formula =}\NormalTok{ y}\OperatorTok{~}\NormalTok{x, }\DataTypeTok{method=}\NormalTok{glm) }\OperatorTok{+}
\StringTok{  }\KeywordTok{labs}\NormalTok{(}\DataTypeTok{title=}\StringTok{"Relación Displacement-Weight"}\NormalTok{) }\OperatorTok{+}
\StringTok{  }\KeywordTok{theme_light}\NormalTok{()}
\end{Highlighting}
\end{Shaded}

\begin{center}\includegraphics{EDA_files/figure-latex/unnamed-chunk-21-2} \end{center}

Negativas

\begin{Shaded}
\begin{Highlighting}[]
\KeywordTok{ggplot}\NormalTok{(auto, }\KeywordTok{aes}\NormalTok{(}\DataTypeTok{x=}\NormalTok{Displacement, }\DataTypeTok{y=}\NormalTok{Mpg)) }\OperatorTok{+}
\StringTok{  }\KeywordTok{geom_point}\NormalTok{() }\OperatorTok{+}
\StringTok{  }\KeywordTok{geom_smooth}\NormalTok{(}\DataTypeTok{formula =}\NormalTok{ y}\OperatorTok{~}\KeywordTok{log}\NormalTok{(x), }\DataTypeTok{method=}\NormalTok{glm) }\OperatorTok{+}
\StringTok{  }\KeywordTok{labs}\NormalTok{(}\DataTypeTok{title=}\StringTok{"Relación Displacement-Mpg"}\NormalTok{) }\OperatorTok{+}
\StringTok{  }\KeywordTok{theme_light}\NormalTok{()}
\end{Highlighting}
\end{Shaded}

\begin{center}\includegraphics{EDA_files/figure-latex/unnamed-chunk-22-1} \end{center}

\begin{Shaded}
\begin{Highlighting}[]
\KeywordTok{ggplot}\NormalTok{(auto, }\KeywordTok{aes}\NormalTok{(}\DataTypeTok{x=}\NormalTok{Weight, }\DataTypeTok{y=}\NormalTok{Mpg)) }\OperatorTok{+}
\StringTok{  }\KeywordTok{geom_point}\NormalTok{() }\OperatorTok{+}
\StringTok{  }\KeywordTok{geom_smooth}\NormalTok{(}\DataTypeTok{formula =}\NormalTok{ y}\OperatorTok{~}\KeywordTok{log}\NormalTok{(x), }\DataTypeTok{method=}\NormalTok{glm) }\OperatorTok{+}
\StringTok{  }\KeywordTok{labs}\NormalTok{(}\DataTypeTok{title=}\StringTok{"Relación Weight-Mpg"}\NormalTok{) }\OperatorTok{+}
\StringTok{  }\KeywordTok{theme_light}\NormalTok{()}
\end{Highlighting}
\end{Shaded}

\begin{center}\includegraphics{EDA_files/figure-latex/unnamed-chunk-22-2} \end{center}

Previsualicación de las variables respecto a la salida

\begin{Shaded}
\begin{Highlighting}[]
\KeywordTok{ggplot}\NormalTok{(}\KeywordTok{melt}\NormalTok{(auto, }\StringTok{"Mpg"}\NormalTok{), }\KeywordTok{aes}\NormalTok{(}\DataTypeTok{x=}\NormalTok{value, }\DataTypeTok{y=}\NormalTok{Mpg, }\DataTypeTok{color=}\NormalTok{variable)) }\OperatorTok{+}
\StringTok{  }\KeywordTok{geom_point}\NormalTok{(}\DataTypeTok{alpha=}\FloatTok{0.3}\NormalTok{) }\OperatorTok{+}
\StringTok{  }\KeywordTok{facet_wrap}\NormalTok{(.}\OperatorTok{~}\NormalTok{variable, }\DataTypeTok{scale=}\StringTok{"free"}\NormalTok{) }\OperatorTok{+}
\StringTok{  }\KeywordTok{labs}\NormalTok{(}\DataTypeTok{title=}\StringTok{"Relación de cada variable respecto de Mpg"}\NormalTok{, }\DataTypeTok{x=}\StringTok{""}\NormalTok{) }\OperatorTok{+}
\StringTok{  }\KeywordTok{theme_light}\NormalTok{()}
\end{Highlighting}
\end{Shaded}

\begin{center}\includegraphics{EDA_files/figure-latex/unnamed-chunk-23-1} \end{center}

Se aprecia alta correlación entre Displacement, Horse\_power, Weight
respecto de la salida.

Como habíamos supuesto en la hipótesis H.9, Horse\_power podría depender
de Displacement y Weight. Esta claro que la potencia de un motor va a
depender de la cilindrada y el peso que tenga.

\begin{Shaded}
\begin{Highlighting}[]
\NormalTok{auto }\OperatorTok
\StringTok{  }\NormalTok{dplyr}\OperatorTok{::}\KeywordTok{select}\NormalTok{(Displacement, Horse_power, Weight) }\OperatorTok\StringTok{  }
\StringTok{  }\KeywordTok{scatterplotMatrix}\NormalTok{(}\DataTypeTok{pch=}\DecValTok{20}\NormalTok{, }\DataTypeTok{col=}\StringTok{"deepskyblue"}\NormalTok{)}
\end{Highlighting}
\end{Shaded}

\begin{center}\includegraphics{EDA_files/figure-latex/unnamed-chunk-24-1} \end{center}

Podemos apreciar como la función de densidad de Horse\_power parece una
(``MEDIANIZACIÓN'') de las otras dos.

Vamos a intentar comprobarlo

\begin{Shaded}
\begin{Highlighting}[]
\KeywordTok{scale}\NormalTok{(auto) }\OperatorTok
\StringTok{  }\KeywordTok{as.data.frame}\NormalTok{() }\OperatorTok\StringTok{ }
\StringTok{  }\KeywordTok{mutate}\NormalTok{(}\DataTypeTok{hp =}\NormalTok{ (Displacement}\OperatorTok{+}\NormalTok{Weight) }\OperatorTok{/}\StringTok{ }\DecValTok{2}\NormalTok{) }\OperatorTok\StringTok{ }
\StringTok{  }\NormalTok{dplyr}\OperatorTok{::}\KeywordTok{select}\NormalTok{(Horse_power, hp) }\OperatorTok\StringTok{  }
\StringTok{  }\KeywordTok{scatterplotMatrix}\NormalTok{(}\DataTypeTok{pch=}\DecValTok{20}\NormalTok{, }\DataTypeTok{col=}\StringTok{"deepskyblue"}\NormalTok{)}
\end{Highlighting}
\end{Shaded}

\begin{center}\includegraphics{EDA_files/figure-latex/unnamed-chunk-25-1} \end{center}

Viendo que no son tan similares como creíamos, buscamos diferentes
fórmulas para el cálculo de los caballos de vapor, y vemos que las
fórmulas son un poco más complejas y no tenemos exactamente los datos
necesarios para utilizarlas (no se descarta que no se puedan deducir,
pero no sería un cálculo evidente)

(poner fórmulas
\url{https://www.ajdesigner.com/phphorsepower/horsepower_equation_trap_speed_method_increase_horsepower.php\#}:\textasciitilde:text=Solving\%20for\%20the\%20change\%20in,the\%20vehicle\%2C\%20driver\%20and\%20passenger.)

\begin{center}\rule{0.5\linewidth}{0.5pt}\end{center}

\hypertarget{tratamiento-de-variables}{%
\subsubsection{Tratamiento de
variables}\label{tratamiento-de-variables}}

Para este dataset, al ser casi todas las variables numéricas continuas,
existen pocos tratamientos que aplicar.

No tenemos variables categóricas que transformar.

Para añadir interpretabilidad, podríamos agrupar la variable Weight en
intervalos, pero puesto que vamos a aplicar regresión sería más
conveniente realizarlo con los resultados finales.

\begin{center}\rule{0.5\linewidth}{0.5pt}\end{center}

\hypertarget{ordenaciones}{%
\subsubsection{Ordenaciones}\label{ordenaciones}}

Volvemos a mostrar la cabecera de los datos:

\begin{Shaded}
\begin{Highlighting}[]
\KeywordTok{head}\NormalTok{(auto)}
\end{Highlighting}
\end{Shaded}

\begin{tabular}{r|r|r|r|r|r}
\hline
Displacement & Horse\_power & Weight & Acceleration & Model\_year & Mpg\\
\hline
91 & 70 & 1955 & 20.5 & 71 & 26.0\\
\hline
232 & 100 & 2789 & 15.0 & 73 & 18.0\\
\hline
350 & 145 & 4055 & 12.0 & 76 & 13.0\\
\hline
318 & 140 & 4080 & 13.7 & 78 & 17.5\\
\hline
113 & 95 & 2372 & 15.0 & 70 & 24.0\\
\hline
97 & 60 & 1834 & 19.0 & 71 & 27.0\\
\hline
\end{tabular}

En este caso no es necesario aplicar ninguna reorganización. Cada
variable ocupa su propia columna, y contiene un único tipo de
información, con unidades de observación diferentes No existe ninguna
relación entre variables sobre la información que codifican (en el
sentido de que podrían agruparse).

\begin{center}\rule{0.5\linewidth}{0.5pt}\end{center}

\hypertarget{resoluciuxf3n-de-hipuxf3tesis}{%
\paragraph{Resolución de
hipótesis}\label{resoluciuxf3n-de-hipuxf3tesis}}

Nos habíamos planteado las siguientes hipótesis

\begin{itemize}
\tightlist
\item
  H.1: Horse\_power puede influir en Mpg: A más potencia, más consumo.
\end{itemize}

\begin{Shaded}
\begin{Highlighting}[]
\KeywordTok{ggplot}\NormalTok{(auto, }\KeywordTok{aes}\NormalTok{(}\DataTypeTok{x=}\NormalTok{Horse_power, }\DataTypeTok{y=}\NormalTok{Mpg)) }\OperatorTok{+}
\StringTok{  }\KeywordTok{geom_point}\NormalTok{() }\OperatorTok{+}
\StringTok{  }\KeywordTok{geom_smooth}\NormalTok{(}\DataTypeTok{formula =}\NormalTok{ y}\OperatorTok{~}\KeywordTok{log}\NormalTok{(x), }\DataTypeTok{method=}\NormalTok{glm) }\OperatorTok{+}
\StringTok{  }\KeywordTok{labs}\NormalTok{(}\DataTypeTok{title=}\StringTok{"Relación Horse_power-Mpg"}\NormalTok{) }\OperatorTok{+}
\StringTok{  }\KeywordTok{theme_light}\NormalTok{()}
\end{Highlighting}
\end{Shaded}

\begin{center}\includegraphics{EDA_files/figure-latex/unnamed-chunk-27-1} \end{center}

Con el plot y los resultados de la matriz de correlación queda claro que
existe una correlación negativa entre estas dos variables. Por tanto,
podemos considerar Horse\_power como un buen candidato para la regresión

\begin{itemize}
\item
  H.2: Weight debe influir en Mpg: Un coche más pesado debería consumir
  más idem. a la hipótesis anterior, lo hemos visto anteriormente en la
  figura X
\item
  H.3: Debería haber correlación entre displacement (cilindrada) con
  horse y acceleration La hemos referenciado anteriormente
\item
  H.4: Horse y acceleration podrían estar relacionadas
\end{itemize}

\begin{Shaded}
\begin{Highlighting}[]
\KeywordTok{ggplot}\NormalTok{(auto, }\KeywordTok{aes}\NormalTok{(}\DataTypeTok{x=}\NormalTok{Horse_power, }\DataTypeTok{y=}\NormalTok{Acceleration)) }\OperatorTok{+}
\StringTok{  }\KeywordTok{geom_point}\NormalTok{() }\OperatorTok{+}
\StringTok{  }\KeywordTok{geom_smooth}\NormalTok{(}\DataTypeTok{formula =}\NormalTok{ y}\OperatorTok{~}\KeywordTok{log}\NormalTok{(x), }\DataTypeTok{method=}\NormalTok{glm) }\OperatorTok{+}
\StringTok{  }\KeywordTok{labs}\NormalTok{(}\DataTypeTok{title=}\StringTok{"Relación Horse_power-Acceleration"}\NormalTok{) }\OperatorTok{+}
\StringTok{  }\KeywordTok{theme_light}\NormalTok{()}
\end{Highlighting}
\end{Shaded}

\begin{center}\includegraphics{EDA_files/figure-latex/unnamed-chunk-28-1} \end{center}

idem. se aprecia una correlación logarítmica entre las dos variables.
Similarmente a lo ocurrido con la hipótesis anterior, esto puede ser un
problema para nuestro problema de regresión.

\begin{itemize}
\tightlist
\item
  H.5: Viendo que contamos con un rango pequeño de años, no debería
  haber un cambio significativo de prestaciones entre años.
\end{itemize}

\begin{Shaded}
\begin{Highlighting}[]
\KeywordTok{ggplot}\NormalTok{(}\KeywordTok{melt}\NormalTok{(auto, }\StringTok{"Model_year"}\NormalTok{), }\KeywordTok{aes}\NormalTok{(}\DataTypeTok{y=}\NormalTok{value, }\DataTypeTok{x=}\NormalTok{Model_year, }\DataTypeTok{color=}\NormalTok{variable)) }\OperatorTok{+}
\StringTok{  }\KeywordTok{geom_point}\NormalTok{(}\DataTypeTok{alpha=}\FloatTok{0.3}\NormalTok{) }\OperatorTok{+}
\StringTok{  }\KeywordTok{facet_wrap}\NormalTok{(.}\OperatorTok{~}\NormalTok{variable, }\DataTypeTok{scale=}\StringTok{"free"}\NormalTok{) }\OperatorTok{+}
\StringTok{  }\KeywordTok{labs}\NormalTok{(}\DataTypeTok{title=}\StringTok{"Plot de cada variable respecto a Model_year"}\NormalTok{) }\OperatorTok{+}
\StringTok{  }\KeywordTok{theme_light}\NormalTok{()}
\end{Highlighting}
\end{Shaded}

\begin{center}\includegraphics{EDA_files/figure-latex/unnamed-chunk-29-1} \end{center}

Existe una alta dispersión de los datos en cada una de las variables,
pero aún así se aprecia tendencias en las variables. Acceleartion y Mpg
tienden a aumentar, y Displacement, Horse\_power y Weight tienden a
disminuir. También vemos que la dispersión en las prestaciones de los
coches disminuyen ligeramente.

Podemos creer en principio que puede deberse a un decremento del número
de instancias con el paso de los años, pero recordamos que en general
los datos están repartidos equitativamente

\begin{Shaded}
\begin{Highlighting}[]
\KeywordTok{table}\NormalTok{(auto}\OperatorTok{$}\NormalTok{Model_year)}
\end{Highlighting}
\end{Shaded}

\begin{verbatim}

70 71 72 73 74 75 76 77 78 79 80 81 82 
29 27 28 40 26 30 34 28 36 29 27 28 30 
\end{verbatim}

Podemos ver cómo varían los rangos para cada año

\begin{Shaded}
\begin{Highlighting}[]
\NormalTok{years <-}\StringTok{ }\NormalTok{auto }\OperatorTok\StringTok{ }\KeywordTok{group_split}\NormalTok{(Model_year)}

\ControlFlowTok{for}\NormalTok{ (y }\ControlFlowTok{in}\NormalTok{ years) \{}
  \KeywordTok{cat}\NormalTok{(}\StringTok{"Year: "}\NormalTok{)}
\NormalTok{  y}\OperatorTok{$}\NormalTok{Model_year[}\DecValTok{1}\NormalTok{] }\OperatorTok\StringTok{ }\KeywordTok{cat}\NormalTok{()}
\NormalTok{  y }\OperatorTok\StringTok{ }\KeywordTok{apply}\NormalTok{(}\DecValTok{2}\NormalTok{, range) }\OperatorTok\StringTok{ }\KeywordTok{as.data.frame}\NormalTok{() }\OperatorTok\StringTok{ }\KeywordTok{print}\NormalTok{()}
\NormalTok{\}}
\end{Highlighting}
\end{Shaded}

\begin{verbatim}
Year: 70  Displacement Horse_power Weight Acceleration Model_year Mpg
1           97          46   1835          8.0         70   9
2          455         225   4732         20.5         70  27
Year: 71  Displacement Horse_power Weight Acceleration Model_year Mpg
1           71          60   1613         11.5         71  12
2          400         180   5140         20.5         71  35
Year: 72  Displacement Horse_power Weight Acceleration Model_year Mpg
1           70          54   2100         11.0         72  11
2          429         208   4633         23.5         72  28
Year: 73  Displacement Horse_power Weight Acceleration Model_year Mpg
1           68          46   1867          9.5         73  11
2          455         230   4997         21.0         73  29
Year: 74  Displacement Horse_power Weight Acceleration Model_year Mpg
1           71          52   1649         13.5         74  13
2          350         150   4699         21.0         74  32
Year: 75  Displacement Horse_power Weight Acceleration Model_year Mpg
1           90          53   1795         11.5         75  13
2          400         170   4668         21.0         75  33
Year: 76  Displacement Horse_power Weight Acceleration Model_year Mpg
1           85          52   1795         12.0         76  13
2          351         180   4380         22.2         76  33
Year: 77  Displacement Horse_power Weight Acceleration Model_year Mpg
1           79          58   1825         11.1         77  15
2          400         190   4335         19.0         77  36
Year: 78  Displacement Horse_power Weight Acceleration Model_year  Mpg
1           78          48   1800         11.2         78 16.2
2          318         165   4080         21.5         78 43.1
Year: 79  Displacement Horse_power Weight Acceleration Model_year  Mpg
1           85          65   1915         11.3         79 15.5
2          360         155   4360         24.8         79 37.3
Year: 80  Displacement Horse_power Weight Acceleration Model_year  Mpg
1           70          48   1845         11.4         80 19.1
2          225         132   3381         23.7         80 46.6
Year: 81  Displacement Horse_power Weight Acceleration Model_year  Mpg
1           79          58   1755         12.6         81 17.6
2          350         120   3725         20.7         81 39.1
Year: 82  Displacement Horse_power Weight Acceleration Model_year Mpg
1           91          52   1965         11.6         82  22
2          262         112   3015         24.6         82  44
\end{verbatim}

\begin{itemize}
\tightlist
\item
  H.6: Pero debería existir una tendencia de mejora de prestaciones con
  los años, incluyendo aumento de Displacement, Horse\_power y
  Acceleration.
\end{itemize}

Ciertamente. Se ha comprobado en la hipótesis anterior.

\begin{itemize}
\tightlist
\item
  H.7: Model\_year podría no mostrar relación con Mpg: Pese al paso de
  los años si contamos con diferentes tipos de vehículos (todoterrenos,
  familiares, deportivos\ldots) podría haber un consumo dispar. (Si
  existiera tendencia, viendo que los años son de las últimas décadas
  del siglo XX, podría ir el consumo hacia abajo)
\end{itemize}

Hemos visto que existe tendencia, lineal con gran dispersión, y
positiva.

\begin{Shaded}
\begin{Highlighting}[]
\KeywordTok{ggplot}\NormalTok{(auto, }\KeywordTok{aes}\NormalTok{(}\DataTypeTok{x=}\NormalTok{Model_year, }\DataTypeTok{y=}\NormalTok{Mpg)) }\OperatorTok{+}
\StringTok{  }\KeywordTok{geom_point}\NormalTok{() }\OperatorTok{+}
\StringTok{  }\KeywordTok{geom_smooth}\NormalTok{(}\DataTypeTok{formula =}\NormalTok{ y}\OperatorTok{~}\NormalTok{x, }\DataTypeTok{method=}\NormalTok{glm) }\OperatorTok{+}
\StringTok{  }\KeywordTok{labs}\NormalTok{(}\DataTypeTok{title=}\StringTok{"Relación Model_year-Mpg"}\NormalTok{) }\OperatorTok{+}
\StringTok{  }\KeywordTok{theme_light}\NormalTok{()}
\end{Highlighting}
\end{Shaded}

\begin{center}\includegraphics{EDA_files/figure-latex/unnamed-chunk-32-1} \end{center}

Por desgracia no contamos información sobre los modelos de los coches

Podemos ver como se ubican los diferentes años en un plot Horse\_power
vs Mpg

\begin{Shaded}
\begin{Highlighting}[]
\KeywordTok{ggplot}\NormalTok{(auto, }\KeywordTok{aes}\NormalTok{(}\DataTypeTok{x=}\NormalTok{Horse_power, }\DataTypeTok{y=}\NormalTok{Mpg, }\DataTypeTok{color=}\NormalTok{Model_year)) }\OperatorTok{+}
\StringTok{  }\KeywordTok{geom_point}\NormalTok{() }\OperatorTok{+}
\StringTok{  }\KeywordTok{geom_smooth}\NormalTok{(}\DataTypeTok{formula =}\NormalTok{ y}\OperatorTok{~}\KeywordTok{I}\NormalTok{(}\KeywordTok{log}\NormalTok{(x)), }\DataTypeTok{color=}\StringTok{"red"}\NormalTok{, }\DataTypeTok{method=}\NormalTok{glm) }\OperatorTok{+}
\StringTok{  }\KeywordTok{labs}\NormalTok{(}\DataTypeTok{title=}\StringTok{"Relación Horse_power-Mpg"}\NormalTok{) }\OperatorTok{+}
\StringTok{  }\KeywordTok{theme_light}\NormalTok{()}
\end{Highlighting}
\end{Shaded}

\begin{center}\includegraphics{EDA_files/figure-latex/unnamed-chunk-33-1} \end{center}

Y vemos que no se puede afirmar la hipótesis, los coches están
entremezclados por diferentes años

\begin{itemize}
\tightlist
\item
  H.8: Esta última hipótesis se puede aplicar al resto de variables,
  indicándonos que Model\_year no debería tener relevancia para este
  problema de regresión.
\end{itemize}

No podemos afirmar la hipótesis anterior y por consiguiente esta
tampoco.

\begin{itemize}
\tightlist
\item
  H.9: Horse\_power podría depender de las variables Displacement y
  Weight
\end{itemize}

Lo hemos comentado anteriormente

\begin{center}\rule{0.5\linewidth}{0.5pt}\end{center}

\hypertarget{conclusiones}{%
\paragraph{Conclusiones}\label{conclusiones}}

Como conclusiones podemos decir que tenemos un dataset altamente
correlacionado, distribuído de forma no normal pero con la información
bien representada. Existen relaciones fuertes entre las variables de
entrada y de las de salida para la regresión que probablemente nos
ayuden a solucionar con facilidad el problema.

Aunque no hemos descubierto los tipos de distribución que siguen
nuestras variables, por si quisiéramos transformarlas a una normal,
podemos sin ninguna duda aplicar una estandarización de los datos
(puesto que sabemos que no afecta negativamente al problema de
regresión) siempre y cuando lo tengamos en cuenta a la hora de analizar
los resultados.

Se nos pide elegir 5 regresores para la regresión y contamos exactamente
con ese número, por lo que no podemos descartar ninguna variable. Aún
así, hemos visto que tenemos algunas variables más interesentas que
otras. Varibles correladas con la salida nos aumentan las posibilidades
de obtener un buen regresor, pero debemos evitar usar variables
correladas entre sí para evitar la multicolinealidad. Sería conveniente
evitarla para aumentar la interpretabilidad del modelo, pero la potencia
en sí de este no cambia.
(\url{https://statisticsbyjim.com/regression/multicollinearity-in-regression-analysis/\#}:\textasciitilde:text=Multicollinearity\%20occurs\%20when\%20independent\%20variables,model\%20and\%20interpret\%20the\%20results.)
(referenciar esta frase en el apartado de regresión)

\end{document}
