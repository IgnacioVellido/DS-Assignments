\section{Análisis de resultados}

En esta sección se pretende analizar las resultados obtenidos más relevantes. Las tablas completas de experimentos se encuentran en la sección \textit{Tablas de resultados}.

% Hay que describir detalladamente todo el proceso algorítmico utilizado, mostrando los resultados de cada uno de los algoritmos utilizados para entrenamiento y test, analizando el comportamiento, y mostrando los flujos/combinaciones de algoritmos de preprocesamiento

Por un lado tenemos que la técnica SSMA sobrepasa el límite de 4GB de memoria impuesto en la práctica, pero en base a las dos ejecuciones con las que contamos (de los primeros días cuando el límite estaba en 46GB) vemos que la reducción en el número de instancias es extremadamente grande, llegando a obtener subconjuntos de 12.000 y 20.000 instancias.
En contraposición vemos que la reducción aplicada por FCNN es significativa, pero de mayor escala en todos sus casos.

\begin{table}[]
    \begin{tabular}{|l|c|c|c|c|r|}
    \hline
    \textbf{Algoritmo} & \textbf{Selección de características} & \textbf{Under/Over-sampling} & \multicolumn{1}{l|}{\textbf{Filtrado de ruido}} & \multicolumn{1}{l|}{\textbf{Selección de instancias}} & \textbf{TPR x TNR} \\ \hline
    Decision Tree      & N                                     & N                            & N                                               & N                                                     & 0.0                \\ \hline
    Random Forest      & N                                     & N                            & N                                               & N                                                     & 0.0                \\ \hline
    PCARD              & No                                    & RUS                          & No                                              & No                                                    & 0.597              \\ \hline
    kNN-IS             & N                                     & No                           & N                                               & N                                                     & 0.0                \\ \hline
    \end{tabular}
    \caption{Flujo de preprocesamiento para los mejores resultados de cada algoritmo.}
\end{table}