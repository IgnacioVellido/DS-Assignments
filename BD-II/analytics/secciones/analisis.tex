\section{Análisis de resultados}

En esta sección se pretende analizar aquellos resultados obtenidos más relevantes. Algunos argumentos también se sustentan sobre las tablas completas de experimentos (con información adicional sobre ellos) que se encuentran en la sección \textit{Tablas de resultados}.

% Hay que describir detalladamente todo el proceso algorítmico utilizado, mostrando los resultados de cada uno de los algoritmos utilizados para entrenamiento y test, analizando el comportamiento, y mostrando los flujos/combinaciones de algoritmos de preprocesamiento

\vspace{\baselineskip}

\begin{table}[H]
    \centering
    \begin{tabular}{|l|c|c|c|c|r|}
    \hline
    \textbf{Algoritmo} & \textbf{\shortstack{Selección de \\ características}} & \textbf{\shortstack{Under/Over \\ sampling}} & \multicolumn{1}{l|}{\textbf{\shortstack{Filtrado \\ de ruido}}} & \multicolumn{1}{l|}{\textbf{\shortstack{Selección de \\ instancias}}} & \textbf{\shortstack{TPR x \\ TNR}} \\ \hline
    Decision Tree      & No & RUS  & HME & FCNN  & 0.606 \\ \hline
    Random Forest      & No & RUS  & HME & FCNN  & 0.607 \\ \hline
    PCARD              & -  & RUS  & No  & No    & 0.598 \\ \hline
    kNN-IS             & No & RUS  & HME & No    & 0.526 \\ \hline
    \end{tabular}
    \caption{Flujo de preprocesamiento para los mejores resultados de cada algoritmo tras la optimización de parámetros.}
    \label{final}
\end{table}


% \subsection{Conclusiones generales}
% Objetivamente la calidad de los resultados es bastante superior a la baseline (cuánto ?), pero no por ello un tanto bajas. La práctica se ha enfocado en evaluar las diferentes combinaciones de técnicas (independientemente del resultado obtenido ?REALLY?). Una posible forma de mejorar sería un ajuste mayor de los hiperparámetros ???

\subsection{Sobre las técnicas de aprendizaje}

En términos de los algoritmos de clasificación, tal y como se muestra en la tabla \ref{final}, obtenemos prácticamente la misma calidad con cualquiera de ellos (variando en las milésimas), siendo ligeramente superior Random Forest y Árboles de Decisión. Para ambas técnicas el flujo de preprocesamiento coincide, reduciendo el tamaño del conjunto de datos a un 10\% del tamaño original.

Mediante las tablas \ref{dt} y \ref{rf} vemos que independientemente del preprocesamiento los resultados son muy similares en los dos casos, probablemente debido al estar un algoritmo formado como ensamblado del otro. A pesar de eso, en media vemos RF ser más robusto con RUS mientras que DT funciona bien con ROS.

Sobre PCARD, aunque se alcanza el máximo con las mínimas técnicas de preprocesamiento (ajustar el desbalanceo es imprescindible en este problema, y los resultados lo demuestran) no llega a ser malo
% PONER LA VARIANZA

Respecto a kNN, notamos peor calidad independientemente de la técnica y parámetros con los que se ha probado. 
% Fijándonos en que obtiene una precisión de 0.887 cuando no se aplica ningún preprocesamiento, deducimos que existen una buena cantidad de puntos entremezclados en el espacio - MEJOR NO PONER

\subsection{Sobre las técnicas de selección de características}

Como se dijo anteriormente, hemos aplicado PCA únicamente a los algoritmos donde tiene sentido, pero a partir de las tablas vemos que los resultados empeoran tras su uso.

Hacemos notar que PCARD se comporta mejor que los árboles de decisión con PCA, a pesar de acabar teniendo un flujo similar. El razonamiento lo achacamos a la discretización aleatoria (RD) de PCARD, que elija el tamaño de los intervalos de manera más inteligente que el de 32 con el que se han entrenado los árboles.

A pesar de todo, podríamos considerar si la reducción de dimensionalidad conseguida es aceptable a costa de la cantidad de empeoramiento obtenida. En este problema, pasando de una media de 0.593 a 0.584, que corresponde a una clasificación errónea de $593000 - 584000 = 9000$ instancias más, dada la semántica del problema no parece una pérdida substancial (si en otro caso tratáramos con un problema médico habría que considerar independientemente el TPR y el TNR antes de tomar esta decisión).

\subsection{Sobre las técnicas de balanceo de datos}

No solo RUS ayuda a acelerar la tarea de aprendizaje, también nos da los mejores resultados. En media vemos que se comporta peor que ROS, debido probablemente a que su combinación con técnicas de selección de instancias reduce en algunos casos de manera excesiva el conjunto de datos.

\subsection{Sobre las técnicas de reducción de ruido}

Respecto al filtrado de ruido, los resultados dan a entender de que el dataset no es de por sí bastante ruidoso, y el posible ruido introducido por las otras técnicas no influyente. No por ello vemos que HME ayuda en la obtención de los mejores valores de TPR x TNR y funciona mejor en este conjunto de datos que NCNEdit

\subsection{Sobre las técnicas de reducción de instancias}

Vemos que el uso de FCNN apenas altera los resultados, pero no por ello deja de ser útil, pues en reduce en torno al 50\% el conjunto de datos, y una situación de big data como la que nos encontramos esto es totalmente deseable, ya que reducimos tiempo de cómputo y carga en el sistema.

Finalmente, indicamos que aunque la técnica SSMA sobrepasa el límite de 4GB de memoria impuesto en la práctica, en base a las dos ejecuciones con las que contamos (de los primeros días cuando el límite estaba en 46GB) vemos que la reducción en el número de instancias es extremadamente grande, llegando a obtener subconjuntos de 12.000 y 20.000 instancias.
A pesar de ello no se obtienen resultados de relevancia, siendo peores que los conseguidos mediante otras técnicas.

\begin{table}[t]
    \centering
    \begin{tabular}{cc|c|c|c|}
    \cline{3-5}
    \multicolumn{1}{l}{\textbf{}} & \textbf{} & \multicolumn{1}{c|}{\textbf{Average}} & \multicolumn{1}{c|}{\textbf{STD}} & \textbf{Max} \\ \hline
    \multicolumn{1}{|c|}{\multirow{3}{*}{Filtrado de ruido}}       & No        & 0.282  & 0.234
    & 0.565    \\ \cline{2-5} 
    \multicolumn{1}{|c|}{}  & HME       & 0.339   & 0.226    & 0.575        \\ \cline{2-5} 
    \multicolumn{1}{|c|}{}  & NCNEdit   & 0.273   & 0.249    & 0.564        \\ \hline
    \multicolumn{1}{|c|}{\multirow{2}{*}{Selección de instancias}} & No        & 0.321  & 0.244    & 0.575        \\ \cline{2-5} 
    \multicolumn{1}{|c|}{}  & FCNN      & 0.277   & 0.219    & 0.574        \\ \hline
    \multicolumn{1}{|c|}{\multirow{2}{*}{Selección de características}} & No        & 0.328  & 0.214    & 0.593        \\ \cline{2-5} 
    \multicolumn{1}{|c|}{}  & PCA      & 0.252    & 0.225    & 0.584        \\ \hline
    \multicolumn{1}{|c|}{\multirow{3}{*}{Balanceo de datos}}       & No        & 0.070  & 0.092    & 0.208        \\ \cline{2-5} 
    \multicolumn{1}{|c|}{}  & ROS       & 0.415   & 0.229    & 0.521        \\ \cline{2-5} 
    \multicolumn{1}{|c|}{}  & RUS       & 0.373   & 0.210    & 0.575        \\ \hline
    \end{tabular}
    \caption{Media de resultados de las diferentes técnicas de preprocesamiento.}
    \label{avg}
\end{table}

\begin{table}[t]
    \centering
    \begin{tabular}{cc|c|c|c|}
    \cline{3-5}
    \multicolumn{1}{l}{\textbf{}} & \textbf{} & \multicolumn{1}{c|}{\textbf{Average}} & \multicolumn{1}{c|}{\textbf{STD}} & \textbf{Max} \\ \hline
    \multicolumn{1}{|c|}{\multirow{3}{*}{Filtrado de ruido}}       & No        & 0.288 & 0.253
    & 0.589    \\ \cline{2-5} 
    \multicolumn{1}{|c|}{}  & HME       & 0.339 &  0.231
    & 0.593        \\ \cline{2-5} 
    \multicolumn{1}{|c|}{}  & NCNEdit   & 0.292 &  0.255
    & 0.584        \\ \hline
    \multicolumn{1}{|c|}{\multirow{2}{*}{Selección de instancias}} & No        & 0.333  & 0.256
    & 0.597        \\ \cline{2-5} 
    \multicolumn{1}{|c|}{}  & FCNN      & 0.289 &  0.231
    & 0.593        \\ \hline
    \multicolumn{1}{|c|}{\multirow{2}{*}{Selección de características}} & No        & 0.341  &  0.241
    & 0.593        \\ \cline{2-5} 
    \multicolumn{1}{|c|}{}  & PCA      & 0.272  & 0.242
    & 0.584        \\ \hline
    \multicolumn{1}{|c|}{\multirow{3}{*}{Balanceo de datos}}       & No        & 0.066  &  0.070
    & 0.215        \\ \cline{2-5} 
    \multicolumn{1}{|c|}{}  & ROS       & 0.426 &  0.136
    & 0.542        \\ \cline{2-5} 
    \multicolumn{1}{|c|}{}  & RUS       & 0.284 &  0.273
    & 0.593        \\ \hline
    \end{tabular}
    \caption{Efectos de las diferentes técnicas de preprocesamiento para árboles de decisión.}
    \label{dt}
\end{table}

\begin{table}[t]
    \centering
    \begin{tabular}{cc|c|c|c|}
    \cline{3-5}
    \multicolumn{1}{l}{\textbf{}} & \textbf{} & \multicolumn{1}{c|}{\textbf{Average}} & \multicolumn{1}{c|}{\textbf{STD}} & \textbf{Max} \\ \hline
    \multicolumn{1}{|c|}{\multirow{3}{*}{Filtrado de ruido}}       & No        & 0.239  & 0.250
    & 0.583    \\ \cline{2-5} 
    \multicolumn{1}{|c|}{}  & HME       & 0.294  & 0.238
    & 0.587        \\ \cline{2-5} 
    \multicolumn{1}{|c|}{}  & NCNEdit   & 0.222  & 0.252
    & 0.585        \\ \hline
    \multicolumn{1}{|c|}{\multirow{2}{*}{Selección de instancias}} & No        & 0.278   & 0.257
    & 0.585        \\ \cline{2-5} 
    \multicolumn{1}{|c|}{}  & FCNN      & 0.224  & 0.229
    & 0.587        \\ \hline
    \multicolumn{1}{|c|}{\multirow{2}{*}{Selección de características}} & No        & 0.316  &  0.258
    & 0.587        \\ \cline{2-5} 
    \multicolumn{1}{|c|}{}  & PCA      & 0.187   & 0.211
    & 0.511        \\ \hline
    \multicolumn{1}{|c|}{\multirow{3}{*}{Balanceo de datos}}       & No        & 0.039  & 0.184
    & 0.196        \\ \cline{2-5} 
    \multicolumn{1}{|c|}{}  & ROS       & 0.353  & 0.273
    & 0.532        \\ \cline{2-5} 
    \multicolumn{1}{|c|}{}  & RUS       & 0.362  & 0.179
    & 0.587        \\ \hline
    \end{tabular}
    \caption{Efectos de las diferentes técnicas de preprocesamiento para Random Forest.}
    \label{rf}
\end{table}

\begin{table}[t]
    \centering
    \begin{tabular}{cc|c|c|c|}
    \cline{3-5}
    \multicolumn{1}{l}{\textbf{}} & \textbf{} & \multicolumn{1}{c|}{\textbf{Average}} & \multicolumn{1}{c|}{\textbf{STD}} & \textbf{Max} \\ \hline
    \multicolumn{1}{|c|}{\multirow{3}{*}{Filtrado de ruido}}       & No        & 0.309   & 0.273
    & 0.597        \\ \cline{2-5} 
    \multicolumn{1}{|c|}{}  & HME       & 0.369   &  0.241
    & 0.595        \\ \cline{2-5} 
    \multicolumn{1}{|c|}{}  & NCNEdit   & 0.286   &  0.290
    & 0.593        \\ \hline
    \multicolumn{1}{|c|}{\multirow{2}{*}{Selección de instancias}} & No        & 0.362   & 0.268
    & 0.597        \\ \cline{2-5} 
    \multicolumn{1}{|c|}{}  & FCNN      & 0.281   & 0.252
    & 0.595        \\ \hline
    \multicolumn{1}{|c|}{\multirow{3}{*}{Balanceo de datos}}       & No        & 0.072   & 0.076
    & 0.186        \\ \cline{2-5} 
    \multicolumn{1}{|c|}{}  & ROS       & 0.496   & 0.325
    & 0.542        \\ \cline{2-5} 
    \multicolumn{1}{|c|}{}  & RUS       & 0.397   & 0.220
    & 0.597        \\ \hline
    \end{tabular}
    \caption{Efectos de las diferentes técnicas de preprocesamiento para PCARD.}
    \label{pcard}
\end{table}

\begin{table}[t]
    \centering
    \begin{tabular}{cc|c|c|c|}
    \cline{3-5}
    \multicolumn{1}{l}{\textbf{}} & \textbf{} & \multicolumn{1}{c|}{\textbf{Average}} & \multicolumn{1}{c|}{\textbf{STD}} & \textbf{Max} \\ \hline
    \multicolumn{1}{|c|}{\multirow{3}{*}{Filtrado de ruido}}       & No        & 0.292  & 0.159
    & 0.491    \\ \cline{2-5} 
    \multicolumn{1}{|c|}{}  & HME       & 0.354   & 0.193
    & 0.525        \\ \cline{2-5} 
    \multicolumn{1}{|c|}{}  & NCNEdit   & 0.294  & 0.200
    & 0.492        \\ \hline
    \multicolumn{1}{|c|}{\multirow{2}{*}{Selección de instancias}} & No        & 0.313    &  0.195
    & 0.525        \\ \cline{2-5} 
    \multicolumn{1}{|c|}{}  & FCNN      & 0.313   &  0.165
    & 0.521        \\ \hline
    \multicolumn{1}{|c|}{\multirow{2}{*}{Selección de características}} & No        & 0.328    & 0.177
    & 0.525        \\ \cline{2-5} 
    \multicolumn{1}{|c|}{}  & PCA      & 0.299   & 0.188
    & 0.516        \\ \hline
    \multicolumn{1}{|c|}{\multirow{3}{*}{Balanceo de datos}}       & No        & 0.103     &  0.038
    & 0.235        \\ \cline{2-5} 
    \multicolumn{1}{|c|}{}  & ROS       & 0.386  & 0.180
    & 0.432        \\ \cline{2-5} 
    \multicolumn{1}{|c|}{}  & RUS       & 0.448  & 0.167
    & 0.525        \\ \hline
    \end{tabular}
    \caption{Efectos de las diferentes técnicas de preprocesamiento para kNN.}
    \label{knn}
\end{table}