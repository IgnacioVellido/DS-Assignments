\section{Creatividad y liderazgo}

% En estas sesiones se proponen ejercicios (denominados entregables) que deben realizarse y ser entregados en un documento pdf.
% Para la entrega en SWAD (zona mis trabajos), se requiere
% – Extensión mínima: 1 página + portada
% – Extensión máxima: 3 páginas + portada
% – La portada debe indicar el nombre del (los) alumno(s) así como su titulación.
% – Letra del texto: Arial 11, interlineado simple
% – Letra de los títulos: Arial 13 (negrita)

\subsection{Entregable 1}
% De la lista de empresas fracasadas disponible en https://www.cbinsights.com/blog/startup-failure-post-mortem/ seleccionen como mínimo una empresa (puede elegir más) y comente el porqué de su fracaso. Discusión: ¿hay solapamientos o paralelismos?

Se elige la empresa \textit{Starsky Robotics}, centrada en el ámbito de conducción autónoma y, más concretamente, la de camiones de trasporte. Entre los principales motivos del fracaso, se indica la pérdida de financiación una vez que los problemas a la hora de cumplimentar el objetivo de la empresa se volvieron aparentes. A pesar de pequeños éxitos, la lentitud del despliegue operativo de camiones autónomos en empresas con poca base tecnológica supuso un impedimento enorme para la startup.

\vspace{\baselineskip}

Adicionalmente, el proceso repetitivo (pero necesario) de validación del sistema para mantener la seguridad en carretera tiene poco gancho para los inversores, pues retrasa la promesa de beneficios. Por otro lado, la competencia (atraía el ojo público) creando funcionalidades extremadamente modernas, pero con mucho camino para un uso fiable.

\vspace{\baselineskip}

También se achaca el problema de que la empresa surgió demasiado pronto y no alcanzó el momentum necesario para mantenerse en el largo camino que queda para conseguir una conducción 100\% autónoma.

% https://medium.com/starsky-robotics-blog/the-end-of-starsky-robotics-acb8a6a8a5f5

% Title: The End of Starsky Robotics

% Product: Starsky Robotics

% Starsky Robotics, an autonomous trucking tech startup, folded in March 2020. In a blog post, CEO Stefan Seltz-Axmacher outlined the reasons the company had failed, stating:

% Timing, more than anything else, is what I think is to blame for our unfortunate fate. Our approach, I still believe, was the right one but the space was too overwhelmed with the unmet promise of AI to focus on a practical solution. As those breakthroughs failed to appear, the downpour of investor interest became a drizzle.


\subsection{Entregable 2}
% – Listar varias “tecnologías”, escoger al azar una. Escoger al azar sustantivo y un adjetivo (sencillos)
% – Proponer un producto o servicio basándonos en la tupla generada aleatoriamente. Tupla de ejemplo (Drones, brazo, amarillo)

% \begin{itemize}
%     \item Aerogel
%     \item Grafeno
%     \item Laser
%     \item Textiles electrónicos
%     \item Turbina
%     \item Batería
%     \item Televisión
%     \item Escáner
%     \item Radio
%     \item Radar
%     \item Armas
%     \item Satélite
%     \item Vehículo % Transporte
% \end{itemize}

\textbf{Nota}: Me gustó la idea que salió y la he querido desarrollar con el resto de ejercicios de la asignatura.

\vspace{\baselineskip}

Usando la página \url{https://www.palabrasaleatorias.com/} se generan las siguientes tuplas:
\begin{itemize}
    \item (cachorro, Holanda)
    \item (anécdota, bingo)
    \item (vivir, reja)
\end{itemize}

\vspace{\baselineskip}

Propongo el siguiente producto a partir de la primera tupla:

\vspace{\baselineskip}

Las cuarentenas impuestas en los países del norte de Europa (como Holanda, Dinamarca, Suecia\dots) debido a la pandemia actual han generado un aumento repentino en el número de adopciones de perros. 
Una vez pasados los confinamientos y la vuelta al trabajo estos animales se encuentran separados de sus dueños durante la mayor parte del día, afectándo negativamente en su salud mental por la falta de interacción social, sobre todo los cachorros de baja edad.

\vspace{\baselineskip}

Para solucionar este problema proponemos un nuevo tipo de robot cuadrúpedo que se ``comunica'' y juega con perros. Gracias a un buen sistema de computación visual, el robot es capaz de reconocer el estado de ánimo del cachorro y adaptar su comportamiento, cambiando entre diferentes modos de juego y compañía. Adicionalmente, el robot es programable y en base a la edad del cachorro adaptar un comportamiento más materno. También es capaz de almacenar un mapa virtual de la casa y mantener al animal en las habitaciones que desee el dueño.


\subsection{Entregable 3}
% – Haga la prueba de Hersey-Blanchard y comente su resultado.

\subsubsection{Herser-Blanchard test}

Mis respuestas han sido: 1. A 2. A 3. A 4. D 5. C 6. D 7. B 8. A 9. C 10. D 11. B 12. D

Dándome como resultado en amplitud de estilos:
\begin{itemize}
    \item Directivo: 3
    \item Persuasivo: 5
    \item Participativo: 3
    \item Delegativo: 1
\end{itemize}

Y en adaptabilidad de estilos:
\begin{itemize}
    \item Directivo: 0
    \item Persuasivo: -1
    \item Participativo: 2
    \item Delegativo: 18
    \item \textbf{Suma}: 19
\end{itemize}

Siendo más altos de lo que me esperaba en persuasivo, pero me cuadra en participativo. Siento que mi perspectiva es ``si algo funciona bien, mejor dejárlo'' e intentar involucrar a todos y que se sientan contentos (yo incluído). El alto valor en directivo lo achaco a que aunque intentaría ser flexible y amigable, siendo líder soy responsable de mis subordinados, y por ello dependo de sus buenos resultados.

Sobre la adaptabilidad, los resultados me concuerdan ya que considero que cada uno es ``experto'' en lo suyo y por tanto su opinión es tan válida como la del líder. Es este último el que debe escuchar, pero en última instancia valorar y decidir.

\subsubsection{Liderazgo situacional}

% – Escriba una reflexión sobre cómo ha aplicado o echado en falta la aplicación de un liderazgo situacional en su vida profesional (académica en su defecto). Nota: si usted no ha sido jefe/líder, escriba sobre su experiencia como subordinado.

Aunque no está dentro del ámbito profesional me gustaría hablar de mi experiencia personal como líder situacional en mi vida social. 
Para poner en situación, pertenezco a un grupo de 4 amigos con una buena costumbre de elegir cuidadosamente los regalos de cumpleaños. La cosa está en que soy el único del grupo preocupado por prever los regalos con antelación y no acabar entregándolos después de tiempo, por lo que suelo asumir el puesto de organizador todos los años.

\vspace{\baselineskip}

Uno de ellos es menos afán de regalos materiales, y siento un rol más participativo cuando propongo y coordino la realización de regalos hechos por nosotros, intentando que se note la participación y el cariño de cada uno.

El rol delegativo se ve con el segundo de ellos, puesto que al tener una amistad más afín con un amigo y menos con el resto del grupo suele indicarle que investigue sobre posibles ideas.

El liderazgo directivo no lo veo en este contexto, pero por el test de Herser-Blanchard tampoco parece que sea muy frecuente en mi persona.

% Mencionar los cuatro tipos de liderazgo

% Entrega 3, cuestionario en SWAD (hacerlo es opcional)
% La reflexión solo vale. Líder en lo que sea propia (amigos ? trabajos de universidad?)
% Sino subordinado identificando los 4 estilos de liderazgo (Identificarlo siempre)
