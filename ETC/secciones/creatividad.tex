\section{Creatividad y liderazgo}

% En estas sesiones se proponen ejercicios (denominados entregables) que deben realizarse y ser entregados en un documento pdf.
% Para la entrega en SWAD (zona mis trabajos), se requiere
% – Extensión mínima: 1 página + portada
% – Extensión máxima: 3 páginas + portada
% – La portada debe indicar el nombre del (los) alumno(s) así como su titulación.
% – Letra del texto: Arial 11, interlineado simple
% – Letra de los títulos: Arial 13 (negrita)

\subsection{Entregable 1}
De la lista de empresas fracasadas disponible en https://www.cbinsights.com/blog/startup-failure-post-mortem/ seleccionen como mínimo una empresa (puede elegir más) y comente el porqué de su fracaso. Discusión: ¿hay solapamientos o paralelismos?

Se elige la empresa \textit{Starsky Robotics}, centrada en el ámbito de conducción autónoma y, más concretamente, la de camiones de trasporte. Entre los principales motivos del fracaso, se indica la pérdida de financiación una vez que los problemas a la hora de cumplimentar el objetivo de la empresa se volvieron aparentes. A pesar de pequeños éxitos, la lentitud del despliegue operativo de camiones autónomos en empresas con poca base tecnológica supuso un impedimento enorme para la startup.

Adicionalmente, el proceso repetitivo (pero necesario) de validación del sistema para mantener la seguridad en carretera tiene poco gancho para los inversores, pues retrasa la promesa de beneficios. Por otro lado, la competencia (atraía el ojo público) creando funcionalidades extremadamente modernas, pero con mucho camino para un uso fiable.

También se achaca el problema de que la empresa surgió demasiado pronto y no alcanzó el momentum necesario para mantenerse en el largo camino que queda para conseguir una conducción 100\% autónoma.

% https://medium.com/starsky-robotics-blog/the-end-of-starsky-robotics-acb8a6a8a5f5

% Title: The End of Starsky Robotics

% Product: Starsky Robotics

% Starsky Robotics, an autonomous trucking tech startup, folded in March 2020. In a blog post, CEO Stefan Seltz-Axmacher outlined the reasons the company had failed, stating:

% Timing, more than anything else, is what I think is to blame for our unfortunate fate. Our approach, I still believe, was the right one but the space was too overwhelmed with the unmet promise of AI to focus on a practical solution. As those breakthroughs failed to appear, the downpour of investor interest became a drizzle.


\subsection{Entregable 2}
% – Listar varias “tecnologías”, escoger al azar una. Escoger al azar sustantivo y un adjetivo (sencillos)
% – Proponer un producto o servicio basándonos en la tupla generada aleatoriamente. Tupla de ejemplo (Drones, brazo, amarillo)

\begin{itemize}
    \item Aerogel
    \item Grafeno
    \item Laser
    \item Textiles electrónicos
    \item Turbina
    \item Batería
    \item Televisión
    \item Escáner
    \item Radio
    \item Radar
    \item Armas
    \item Satélite
    \item Vehículo % Transporte
\end{itemize}


\subsection{Entregable 3}
% – Haga la prueba de Hersey-Blanchard y comente su resultado.

% – Escriba una reflexión sobre cómo ha aplicado o echado en falta la aplicación de un liderazgo situacional en su vida profesional (académica en su defecto). Nota: si usted no ha sido jefe/líder, escriba sobre su experiencia como subordinado.

% Desde la perspectiva de un subordinado...