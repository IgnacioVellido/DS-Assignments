\section{Modelo CANVAS}

% Aquí una imagen

% 9 campos del canvas y explicarlos

% Canvas (esquema y describirlo, en presentación quizás)

\subsection{Segmento de clientes}

% SC-­‐ Segmento de clientes: grupos de personas o entidades a las que dirigimos las propuestas de valor. ¿Para quien creamos valor? ¿Nos dirigimos a uno o a diferentes segmentos? (Mercado de masas, nicho de mercado, mercado segmentado) . 

La app se puede dirigir a cualquier grupo de personas que posean un móvil, pero debemos tener en cuenta que el problema que pretende resolver puede ser vergonzoso o ridículo para gran parte de la población.

Por tanto, debemos enfocarnos en diferentes tipos de público objetivo que permitan (volver normal) y arrancar el negocio. Estos son:
\begin{itemize}
    \item \textbf{Turistas}: Estando en una ciudad desconocida y sentir la necesidad puede ser un problema grande. Si con el tiempo es posible alcanzar un rango de efectividad semi-global este sector podrá ser muy prometedor.
    \item \textbf{Trabajadores en la calle}: En este sector englobamos personas tales como taxistas, transportistas (camioneros), y cualquier tipo de persona que se desplace constantemente en sus horarios de trabajo. La probabilidad de uso de la aplicación por parte de este grupo de usuarios es menor pues gran parte de ellos ya conocerán lugares donde ir al servicio. Aunque nuestra app probablemente tengo un uso más esporádico, la inclusión de estimación de ocupación e higiene puede ser útil para este grupo de personas.
    \item \textbf{Usuarios con problemas intestinales}: Principal público objetivo. Al ser conscientes de su enfermedad tendrán hábitos y preocupaciones que los inciten a usar la aplicación, con la idea de tener el máximo de información es sus decisiones (por ej: un usuario puede usar la información para decidir dónde comer si cree que va a tener una urgencia tras la comida).
\end{itemize}

\subsection{Propuesta de valor}

% PV-­‐ Propuestas de valor: productos y servicios que crean valor para un segmento de mercado específico. El obje8vo es solucionar los problemas de los clientes: “Qué quiere comprar nuestro cliente" versus "qué vendemos". 

Proponemos información detallada y a tiempo real de los mejores lugares en los que ir al baño para nuestros usuarios. Para tenemos en cuenta no solo las preferencias del cliente (nivel de urgencia, localización) sino también la calidad de los posibles lugares (higiene, ocupación).

Todo esto además se ofrece de manera gratuita para el usuario.

\subsection{Canales de comunicación, distribución y venta}

% C-­‐ Canales de comunicación, distribución y venta: la forma en que la empresa establece contacto con los diferentes clientes y cómo les proporciona la propuesta de valor. 

% Anuncios en redes sociales, Youtube. Se espera que a la larga la manera que resulte más eficiente sea el boca-a-boca.

% Soporte mediante una página web ?

\subsection{Relación con los clientes}

% RC-­‐ Relación con los clientes: relaciones de la empresa con cada segmento de clientes. En función de cada cliente, adaptaremos el discurso.

\subsection{Ingresos}

% I-­‐Ingresos: se generan cuando los clientes compran las propuestas de valor que ofrece la empresa. ¿Por qué valor pagarían nuestros clientes? ¿Cómo pagan ahora? ¿Cómo les gustaría pagar? 

Se podría considerar añadir información extra tras algunos modelos de pago, pero es esencial que la funcionalidad principal se ofrezca gratuítamente al usuario si queremos alcanzar fidelidad con ellos. Esto nos deja con dos posibles formas de ganar ingresos:

\begin{itemize}
    \item Anuncios de manera no intrusiva, de forma que no empeoremos la opinión de los usuarios.
    \item Venta de datos.
\end{itemize}

\subsection{Recursos y capacidades}

% RC-­‐Recursos y capacidades clave: acFvos necesarios para el modelo de negocio, incluidas las personas de la empresa y sus capacidades (Recursos esicos, intelectuales, humanos, y económicos). 

\begin{itemize}
    \item Servidores y bases de datos que soporten el sistema. Preferiblemente contratado mediante un modelo SaaS.
    \item Personal de desarrollo, soporte, y reparación de la aplicación ante posibles caídas.
    \item Personal legal y de recursos humanos ?
\end{itemize}

\subsection{Actividades clave}

% AC-­‐Ac8vidades clave: acciones necesarias que deben llevarse a cabo si contamos con las capacidades y recursos necesarios (Producción, I+D, Resolución de problemas, Plataforma..) 

\subsection{Partners clave}
% Google y Apple ?

% PC -­‐ Partners (Alianzas) clave: las alianzas, los socios, incluso los proveedores que necesitamos para el éxito del modelo de negocio. Algunas acFvidades se pueden externalizar. 

Se podrían usar mapas de software libre.
En otro caso, si quisiéramos usar las APIs de Google y Apple, deberíamos tener una buena relación con ellos.

\subsection{Estructura de costes}

% EC-­‐ Estructura de costes: gastos asociados a la puesta en marcha de un negocio para poder elaborar y hacer llegar la propuesta de valor a los clientes (Costes fijos, variables, low-­‐cost, según valor, economias de escala,..)

El coste de los recursos tecnológicos tendrá una base fija y una variable en función de la demanda de usuarios y el rango de soporte (países) de la aplicación.
Existe también un coste de distribución en las tiendas de software móvil (App Store y Play Store)

Coste fijo del personal técnico y (legal?)