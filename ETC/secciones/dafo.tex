\section{Análisis DAFO}  

\subsection{Fortalezas}
• Qué ventajas Fene nuestra empresa (o idea inicial) sobre la competencia (técnicas, de costes, de experiencia, de recursos humanos, .. 
• Qué elementos perciben nuestros clientes como una fortaleza de empresa 
• Cual es el know-­‐how de nuestros recursos humanos 
• Donde está nuestra innovación 

\begin{itemize}
    \item Capacidad de crear una aplicación rápida, sencilla, limpia y eficaz que resuelve el problema de la mejor forma posible.
    \item Competencia actual aparentemente inexistente.
\end{itemize}

\subsection{Debilidades}
• Qué hace peor nuestra empresa que la competencia 
• En qué procesos somos más lentos o ineficientes 
• Qué perciben nuestros clientes como debilidades 
• Qué aspectos tecnológicos del sector aún no hemos incorporado 
• Qué nos dificulta adaptarnos a las peFciones de nuestros clientes 

\begin{itemize}
    \item 
\end{itemize}

\subsection{Oportunidades}
• Qué tendencias favorables presenta el mercado 
• Qué necesidades de los clientes no están cubiertas por la competencia 
• Qué cambios legislaFvos posiFvos se han producido o se prevén 
• Qué hábitos de vida o de infraestructuras han cambiado y pueden favorecer al sector 

\begin{itemize}
    \item Problema real y (aparentemente) que no puede desaparecer.
\end{itemize}

\subsection{Amenazas}
• Qué cambios tecnológicos que yo no dispongo están sucediendo en el mercado 
• Qué hábitos de consumo se prevén que puedan reducir el mercado 
• Qué tendencias demográficas pueden perjudicar al sector 
• Cual es la situación del sector financiero 
• Qué acuerdos internacionales pueden perjudicarnos 

\begin{itemize}
    \item Problemas legales con el uso o venta de datos.
    \item Desacuerdo de comerciantes que se opongan a que aparezcan en la aplicación.
    \item Posibilidad de que aparezca competencia extremadamente fuerte (por las empresas con infraestructura y usuarios ya establecidos (Google, Apple)).
\end{itemize}