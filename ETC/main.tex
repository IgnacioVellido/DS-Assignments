\documentclass[13pt,a4paper]{article}
\usepackage[spanish,es-nodecimaldot]{babel}	% Utilizar español
\usepackage[utf8]{inputenc}					% Caracteres UTF-8
\usepackage{graphicx}						% Imagenes
\usepackage[hidelinks]{hyperref}			% Poner enlaces sin marcarlos en rojo
\usepackage{fancyhdr}						% Modificar encabezados y pies de pagina
\usepackage{float}							% Insertar figuras
\usepackage[textwidth=390pt]{geometry}		% Anchura de la pagina
\usepackage[nottoc]{tocbibind}				% Referencias (no incluir num pagina indice en Indice)
\usepackage{enumitem}						% Permitir enumerate con distintos simbolos
\usepackage[T1]{fontenc}					% Usar textsc en sections
\usepackage{amsmath}						% Símbolos matemáticos
\usepackage[ruled,vlined]{algorithm2e}      % Pseudocódigo
\usepackage{xcolor}
\usepackage{listings}
% Para que acepten tíldes los listing
\lstset{     
     literate=%
         {á}{{\'a}}1
         {é}{{\'e}}1
         {í}{{\'i}}1
         {ó}{{\'o}}1
         {ú}{{\'u}}1
         {Á}{{\'A}}1
         {É}{{\'E}}1
         {Í}{{\'I}}1
         {Ó}{{\'O}}1 
         {Ú}{{\'U}}1
         {ñ}{{\~n}}1 
         {Ñ}{{\~N}}1 
         {¿}{{?``}}1 
         {¡}{{!``}}1
}
\usepackage{dsfont}

% ==============================================================================

\usepackage{caption, subcaption}
\usepackage[section]{placeins}
\makeatletter
\def\fps@figure{H}
\makeatother

\usepackage{booktabs}
\usepackage{longtable}
\usepackage{array}
\usepackage{multirow}
\usepackage{wrapfig}
\usepackage{colortbl}
\usepackage{pdflscape}
\usepackage{tabu}
\usepackage{threeparttable}
\usepackage{threeparttablex}
\usepackage[normalem]{ulem}
\usepackage{makecell}
\usepackage{xcolor}
\usepackage[bottom]{footmisc}

\makeatletter
\newcommand*{\centerfloat}{%
  \parindent \z@
  \leftskip \z@ \@plus 1fil \@minus \textwidth
  \rightskip\leftskip
  \parfillskip \z@skip}
\makeatother

% ==============================================================================
% ==============================================================================

% Comando para poner el nombre de la asignatura
\newcommand{\asignatura}{Emprendimiento y transferencia del conocimiento}
\newcommand{\autor}{Ignacio Vellido Expósito}
\newcommand{\email}{ignaciove@correo.ugr.es}
\newcommand{\titulo}{Robot mascota para perros}
\newcommand{\subtitulo}{Trabajos}

% Configuracion de encabezados y pies de pagina
\pagestyle{fancy}
\lhead{\autor{}}
\rhead{\asignatura{}}
\lfoot{Máster Ciencia de Datos e Ingeniería de Computadores}
\cfoot{}
\rfoot{\thepage}
\renewcommand{\headrulewidth}{0.4pt}		% Linea cabeza de pagina
\renewcommand{\footrulewidth}{0.4pt}		% Linea pie de pagina

% ==============================================================================
% ==============================================================================

\usepackage[final]{pdfpages}
\newcommand\invisiblesection[1]{%
  \refstepcounter{section}%
  \addcontentsline{toc}{section}{\protect\numberline{\thesection}#1}%
  \sectionmark{#1}}

\begin{document}
    \pagenumbering{gobble}
    % ==============================================================================
% Pagina de titulo
\begin{titlepage}
    \begin{minipage}{\textwidth}
        \centering

        \includegraphics[scale=0.5]{img/ugr.png}\\

        \textsc{\Large \asignatura{}\\[0.2cm]}
        \textsc{MÁSTER CIENCIA DE DATOS E INGENIERÍA DE COMPUTADORES}\\[1cm]

        \noindent\rule[-1ex]{\textwidth}{1pt}\\[1.5ex]
        \textsc{{\Huge \titulo\\[0.5ex]}}
        \textsc{{\Large \subtitulo\\}}
        \noindent\rule[-1ex]{\textwidth}{2pt}\\[2.5ex]

        \end{minipage}

        \vspace{0.3cm}

        \begin{minipage}{\textwidth}

        \centering

        \textbf{Autor}\\ {\autor{}}\\[1.5ex]
        \vspace{0.4cm}

        \includegraphics[scale=0.3]{img/etsiit.jpeg}
        \includegraphics[scale=0.6]{img/master.png}

        \vspace{0.7cm}
        \textsc{Escuela Técnica Superior de Ingenierías Informática y de Telecomunicación}\\
        \vspace{1cm}
        \textsc{Curso 2020-2021}
    \end{minipage}
\end{titlepage}
% ==============================================================================
    
    \pagenumbering{arabic}
    \tableofcontents
    \thispagestyle{empty}				% No usar estilo en la pagina de indice

    \newpage

    % ==============================================================================

    % •    Desarrollo de una propuesta sencilla de plan inicial (modelo de negocio) utilizando el método CANVAS + DAFO.
    % •    Realización de una ficha de búsqueda de financiación empresarial.
    % •    Ejercicio práctico de búsqueda de patentes.
    % •    Realización en casa de un video individual (“Elevator Pitch” de menos de 5 minutos) grabado por cada estudiante, presentando su idea de negocio.
    % •    Realización de una tabla con previsiones financieras
    % •    Ejercicio de desarrollo de creatividad y de liderazgo.

    \section{Modelo CANVAS}

Robot cuadrúpedo que acompaña, juega y se ``comunica'' con perros.

\begin{figure}[H]
    \centering
    \makebox[\textwidth][c]{\includegraphics[width=1.3\textwidth]{img/canvas.png}}%
\end{figure}

\subsection{Segmento de clientes}

% SC-­‐ Segmento de clientes: grupos de personas o entidades a las que dirigimos las propuestas de valor. ¿Para quien creamos valor? ¿Nos dirigimos a uno o a diferentes segmentos? (Mercado de masas, nicho de mercado, mercado segmentado).

Enfocamos el producto a personas de clase media-alta mayoritariamente con un solo perro, y que por diversos motivos (trabajo, salud\dots) frecuentemente no puedan pasar gran parte del día con sus mascotas.

Estimamos que el mercado objetivo es pequeño, y solo si los costes del producto se consiguen abaratar se podrá expandir a un rango mayor de clientes.

\vspace{\baselineskip}

Adicionalmente, como plan de expansión, se podría intentar aumentar el realismo para que no solo sea creíble para un animal, sino también para un humano. De esta forma podríamos entrar en el mercado de robots de compañía (eso sí, un sector más competitivo).

\subsection{Propuesta de valor}

% PV-­‐ Propuestas de valor: productos y servicios que crean valor para un segmento de mercado específico. El obje8vo es solucionar los problemas de los clientes: “Qué quiere comprar nuestro cliente" versus "qué vendemos". 

Queremos que el cliente sienta que sus mascotas no se encuentran solas durante su ausencia, proponiendo un robot compañero para ellas. Este robot cuadrúpedo, fabricado con una capa interna robusta y una externa realista, es capaz de reconocer el estado de ánimo del animal y adaptar su comportamiento, cambiando entre diferentes modos de juego y compañía.

\subsection{Canales de comunicación, distribución y venta}

% C-­‐ Canales de comunicación, distribución y venta: la forma en que la empresa establece contacto con los diferentes clientes y cómo les proporciona la propuesta de valor. 

Estamos hablando de un producto de lujo donde el público mayoritario (dueños de perros) no es el cliente objetivo (dueños ausentes en su casa). Por tanto creemos que la mejor forma de entrar en contacto será con publicidad dirigida a este segmento y mediante el boca-a-boca.

El canal de distribución principal será online con producción bajo demanda y, en caso de existir en el país donde se despliegue el comercio, tiendas de robótica especializadas.

\subsection{Relación con los clientes}

% RC-­‐ Relación con los clientes: relaciones de la empresa con cada segmento de clientes. En función de cada cliente, adaptaremos el discurso.

Será recomendable ofrecer soporte software a medio/largo plazo del robot, añadiendo nueva funcionalidad y corrigiendo errores. Buscaremos así una fidelidad con los clientes a través de sus mascotas.
Será importante no solo que el robot obtenga el cariño de los perros, sino también del dueño, de forma que vea valor en la adquisición del producto.

\subsection{Ingresos}

% I-­‐Ingresos: se generan cuando los clientes compran las propuestas de valor que ofrece la empresa. ¿Por qué valor pagarían nuestros clientes? ¿Cómo pagan ahora? ¿Cómo les gustaría pagar? 

Los clientes harán un único pago por el producto al completo y su suporte. Para mantener una buena relación con ellos, mejoras de intelecto o actitud que no requieran rediseño del hardware deberían ofrecerse gratuitamente.

\subsection{Recursos y capacidades}

% RC-­‐Recursos y capacidades clave: acFvos necesarios para el modelo de negocio, incluidas las personas de la empresa y sus capacidades (Recursos esicos, intelectuales, humanos, y económicos) 

Tendremos recursos de personal (investigadores, ingenieros, desarrolladores, gestores\dots) e intelectualmente en las patentes que se desarrollen.

\subsection{Actividades clave}

% AC-­‐Ac8vidades clave: acciones necesarias que deben llevarse a cabo si contamos con las capacidades y recursos necesarios (Producción, I+D, Resolución de problemas, Plataforma..)

La empresa debería contar con los siguientes segmentos:
\begin{itemize}
    \item Equipo de I+D buscando mejores revisiones del robot e investigando sobre nuevos posibles productos. Se deberá investigar tanto por la parte hardware como por la software (comportamientos, mejoras en el sistema sensorial, etc.).
    \item Equipo de producción y fabricación.
    \item Equipo de soporte.
    \item Equipo especializado en marketing y en ``hacer llegar'' el producto a los clientes.
\end{itemize}

\subsection{Partners clave}
% PC -­‐ Partners (Alianzas) clave: las alianzas, los socios, incluso los proveedores que necesitamos para el éxito del modelo de negocio. Algunas acFvidades se pueden externalizar. 

Nuestros partners clave serán aquellos que nos proporcionen los componentes básicos para el montaje. Adicionalmente, si queremos que el comportamiento sea reconfigurable, dependeremos de la compañía/comunidad propietaria del sistema software que utilicemos (o sobre la que se base nuestros sistemas inteligentes).

\subsection{Estructura de costes}

% EC-­‐ Estructura de costes: gastos asociados a la puesta en marcha de un negocio para poder elaborar y hacer llegar la propuesta de valor a los clientes (Costes fijos, variables, low-­‐cost, según valor, economias de escala,..)

Tendremos costes fijos de personal en los equipos mencionados anteriormente (I+D, producción, soporte, marketing\dots). También intentaremos mantener costes estables (no variables) con nuestros proveedores. \newpage
    \section{Análisis DAFO}  

\subsection{Fortalezas}
% • Qué ventajas Fene nuestra empresa (o idea inicial) sobre la competencia (técnicas, de costes, de experiencia, de recursos humanos, .. 
% • Qué elementos perciben nuestros clientes como una fortaleza de empresa 
% • Cual es el know-­‐how de nuestros recursos humanos 
% • Donde está nuestra innovación 

Nosotros proponemos un robot cuadrúpedo que haga que la mascota se encuentra con ``un igual'', de forma que juegue y se comunique con el al igual que lo haría con otro perro.


\subsection{Debilidades}
% • Qué hace peor nuestra empresa que la competencia 
% • En qué procesos somos más lentos o ineficientes 
% • Qué perciben nuestros clientes como debilidades 
% • Qué aspectos tecnológicos del sector aún no hemos incorporado 
% • Qué nos dificulta adaptarnos a las peFciones de nuestros clientes 

El producto a desarrollar es extremadamente complejo y costoso. Necesitamos proveer de sistemas sensoriales, motores e intelectuales que ``confundan'' a una máquina con un animal (desde la perspectiva de la mascota en principio). El sistema debe ser robusto y seguro, lo que conlleva mucho tiempo de investigación y testeo.

Todo esto se resume en que contaríamos con un proceso pre-producción largo y caro.

\subsection{Oportunidades}
% • Qué tendencias favorables presenta el mercado 
% • Qué necesidades de los clientes no están cubiertas por la competencia 
% • Qué cambios legislaFvos posiFvos se han producido o se prevén 
% • Qué hábitos de vida o de infraestructuras han cambiado y pueden favorecer al sector 

Los robots enfocados a perros en la actualidad carecen de facultades realistas que permitan que el animal empatice de forma más profunda. La mayoría vagamente pasan del umbral de ``juguetes con ruedas'' y buscan más paliar el servicio de proporcionar comida que de acompañar.

\vspace{\baselineskip}

La masiva adopción de perros por la pandemia en hogares/familias que no estaban preparados para ellos a largo plazo acaba con los animales gran parte del día solos en casa. Ya que tras el levantamiento de los confinamientos los dueños vuelven a la rutina y están poco tiempo presentes en la casa. Nuestro robot mascota aporta entretenimiento y compañía para los animales durante estas horas de ausencia.

% https://www.kickstarter.com/projects/51244428/mia-a-friendly-robot-for-cats-and-dogs-by-kolony-r

\subsection{Amenazas}
% • Qué cambios tecnológicos que yo no dispongo están sucediendo en el mercado 
% • Qué hábitos de consumo se prevén que puedan reducir el mercado 
% • Qué tendencias demográficas pueden perjudicar al sector 
% • Cual es la situación del sector financiero 
% • Qué acuerdos internacionales pueden perjudicarnos 

La posibilidad de abandono de las mascotas en los hogares de los posibles clientes puede reducir significativemente el mercado. Adicionalmente, la gran cantidad de financiación necesaria y lo complejo del desarrollo puede acabar con una falta monetaria antes de sacar el robot en producción. \newpage
    \invisiblesection{Financiación}
    \includepdf[pages=-]{Ficha_Busqueda_Financiacion.pdf}
    \section{Patentes}

\subsection{}

Buscar patentes de relacionadas con el campo de vuestra idea de negocio (por ejemplo patentes relacionadas con redes neuronales, patentes relacionadas con lógica difusa, etc. Si vuestra idea de negocio o tecnología se relaciona con “soft-computing”). Buscar con palabras clave (key words).

\par\noindent\rule{\textwidth}{0.4pt}

He buscado patentes a partir de las keywords ``robot dog''.

\begin{itemize}
    \item ¿Cuántas son?. Utilizando LENS indicar con un gráfico cómo es la evolución de patentes en este campo.
    
    \begin{itemize}
        \item LENS: 20.846 patentes.
        \item Google patents: 75.624 patentes.
        \item Espacenet: 13.335 patentes.
    \end{itemize}
    
    \begin{figure}[H]
        \centering
        \includegraphics[width=.8\textwidth]{img/patentes/2a.png}
    \end{figure}

    Mirando por encima no todas patentan un robot al completo. Muchas de ellas describen sistemas de visión, sistemas de control, circuitos\dots.

    \item Identificar algún código de clasificación de patentes (CPC o CIP) relacionado.
    
    He visto 2 códigos CPC frecuentemente:
    \begin{itemize}
        \item A63H11/00: Derivado de salud y juguetes.
        \item B25J19/00: Para sistema de control de manipuladores.
    \end{itemize}

    \item Indicar número de patentes de ese código (CPC o CIP).
    \begin{itemize}
        \item A63H11/00: 5.396 patentes.
        \item B25J19/00: 48.745 patentes.
    \end{itemize}

    \item Indicar las tres principales empresas que tienen patentes relacionadas con ese código (CPC o CIP).
    \begin{itemize}
        \item A63H11/00: Sony Corp (636), Mattel INC (172), Groove X INC (95).
        \item B25J19/00: Fanuc LTD (1.164), Seiko Epson Corp (590), Kawasaki Heavy Industries (461).
    \end{itemize}

    \item Buscar una patente en concreto e indicar el link donde aparezcan los “claims” (o reivindicaciones) de una patente en este campo.
    
    Patente CN205273661U:

    \url{https://worldwide.espacenet.com/patent/search/family/056058262/publication/CN205273661U?q=pn%3DCN205273661U}
\end{itemize}


\subsection{}

A la hora de valorar patentes se puede tener en cuenta el crecimiento del área tecnológica, que a su vez se puede medir de forma indirecta analizando el crecimiento registrado en el número de solicitudes de patente en un área específica de la tecnología, valorando positivamente aquellas tecnologías cuyas patentes hayan registrado un crecimiento continuado en el pasado reciente (20 años) frente a las que hayan registrado un crecimiento negativo, discontinuo o alejado en el tiempo.

Buscar tendencias de patentes en las siguientes temáticas (utilizar el buscador “LENS”).

Para cada caso añadir el gráfico de tendencia anual de patentes sobre esta temática (gráfico de número de patentes por año relacionadas con ese campo):

\par\noindent\rule{\textwidth}{0.4pt}

\begin{itemize}
    \item Buscar patentes sobre “Face recognition”. Indicar cuántas tiene “Samsung” sobre esta temática

    498.029 patentes. 10.504 de Samsung.

    \begin{figure}[H]
        \centering
        \includegraphics[width=.8\textwidth]{img/patentes/2a.png}
    \end{figure}

    \item Buscar patentes sobre “Fuzzy logic”. Indicar cuántas tiene “Microsoft” sobre esta temática.

    96.435 patentes. 4.696 + 2.141 del conglomerado de empresas de Microsoft.

    \begin{figure}[H]
        \centering
        \includegraphics[width=.8\textwidth]{img/patentes/2b.png}
    \end{figure}

    \item Buscar patentes sobre “SVM” (Support Vector Machine). Indicar cuántas tiene “Microsoft” sobre esta temática.
    
    504.121 patentes. 6.829 + 5.415 del conglomerado de empresas de Microsoft.

    \begin{figure}[H]
        \centering
        \includegraphics[width=.8\textwidth]{img/patentes/2c.png}
    \end{figure}
\end{itemize}

\subsection{}

Búsqueda de una patente y relación con patentes similares. Por ejemplo con Google Patents o Espacenet.

Buscar la patente WO2020033205A1. Indicar: 
\begin{itemize}
    \item \textbf{Los Inventores}: Stephen Alan Mckinley, David Gealy y Pieter Abbeel.
    \item \textbf{Institución o persona que realiza la solicitud}: Universidad de California.
    \item \textbf{Fecha de la solicitud}: 2019-07-31.
    \item \textbf{Fecha de la publicación}: 2020-02-13.
    \item \textbf{Códigos de Clasificación (CPC)}:
    \begin{itemize}
        \item B25J18/00 (US) 
        \item B25J9/0087 (EP)
        \item B25J9/102 (EP)
        \item B25J9/126 (EP,US)
        \item F16H48/38 (US) 
        \item B25J19/06 (US)
        \item F16H2048/387 (US)
    \end{itemize}
\end{itemize}


% Patentes
% 1) Quizás desarrollar un poco
% 2) Solo el número y (opcional) una gráfica
% 3) Solo las preguntas

% https://worldwide.espacenet.com/patent/search/family/069228268/publication/WO2020033205A1?q=WO2020033205A1 \newpage
    \section{Creatividad y liderazgo}

% En estas sesiones se proponen ejercicios (denominados entregables) que deben realizarse y ser entregados en un documento pdf.
% Para la entrega en SWAD (zona mis trabajos), se requiere
% – Extensión mínima: 1 página + portada
% – Extensión máxima: 3 páginas + portada
% – La portada debe indicar el nombre del (los) alumno(s) así como su titulación.
% – Letra del texto: Arial 11, interlineado simple
% – Letra de los títulos: Arial 13 (negrita)

\subsection{Entregable 1}
% De la lista de empresas fracasadas disponible en https://www.cbinsights.com/blog/startup-failure-post-mortem/ seleccionen como mínimo una empresa (puede elegir más) y comente el porqué de su fracaso. Discusión: ¿hay solapamientos o paralelismos?

Se elige la empresa \textit{Starsky Robotics}, centrada en el ámbito de conducción autónoma y, más concretamente, la de camiones de trasporte. Entre los principales motivos del fracaso, se indica la pérdida de financiación una vez que los problemas a la hora de cumplimentar el objetivo de la empresa se volvieron aparentes. A pesar de pequeños éxitos, la lentitud del despliegue operativo de camiones autónomos en empresas con poca base tecnológica supuso un impedimento enorme para la startup.

\vspace{\baselineskip}

Adicionalmente, el proceso repetitivo (pero necesario) de validación del sistema para mantener la seguridad en carretera tiene poco gancho para los inversores, pues retrasa la promesa de beneficios. Por otro lado, la competencia (atraía el ojo público) creando funcionalidades extremadamente modernas, pero con mucho camino para un uso fiable.

\vspace{\baselineskip}

También se achaca el problema de que la empresa surgió demasiado pronto y no alcanzó el momentum necesario para mantenerse en el largo camino que queda para conseguir una conducción 100\% autónoma.

% https://medium.com/starsky-robotics-blog/the-end-of-starsky-robotics-acb8a6a8a5f5

% Title: The End of Starsky Robotics

% Product: Starsky Robotics

% Starsky Robotics, an autonomous trucking tech startup, folded in March 2020. In a blog post, CEO Stefan Seltz-Axmacher outlined the reasons the company had failed, stating:

% Timing, more than anything else, is what I think is to blame for our unfortunate fate. Our approach, I still believe, was the right one but the space was too overwhelmed with the unmet promise of AI to focus on a practical solution. As those breakthroughs failed to appear, the downpour of investor interest became a drizzle.


\subsection{Entregable 2}
% – Listar varias “tecnologías”, escoger al azar una. Escoger al azar sustantivo y un adjetivo (sencillos)
% – Proponer un producto o servicio basándonos en la tupla generada aleatoriamente. Tupla de ejemplo (Drones, brazo, amarillo)

% \begin{itemize}
%     \item Aerogel
%     \item Grafeno
%     \item Laser
%     \item Textiles electrónicos
%     \item Turbina
%     \item Batería
%     \item Televisión
%     \item Escáner
%     \item Radio
%     \item Radar
%     \item Armas
%     \item Satélite
%     \item Vehículo % Transporte
% \end{itemize}

\textbf{Nota}: Me gustó la idea que salió y la he querido desarrollar con el resto de ejercicios de la asignatura.

\vspace{\baselineskip}

Usando la página \url{https://www.palabrasaleatorias.com/} se generan las siguientes tuplas:
\begin{itemize}
    \item (cachorro, Holanda)
    \item (anécdota, bingo)
    \item (vivir, reja)
\end{itemize}

\vspace{\baselineskip}

Propongo el siguiente producto a partir de la primera tupla:

\vspace{\baselineskip}

Las cuarentenas impuestas en los países del norte de Europa (como Holanda, Dinamarca, Suecia\dots) debido a la pandemia actual han generado un aumento repentino en el número de adopciones de perros. 
Una vez pasados los confinamientos y la vuelta al trabajo estos animales se encuentran separados de sus dueños durante la mayor parte del día, afectándo negativamente en su salud mental por la falta de interacción social, sobre todo los cachorros de baja edad.

\vspace{\baselineskip}

Para solucionar este problema proponemos un nuevo tipo de robot cuadrúpedo que se ``comunica'' y juega con perros. Gracias a un buen sistema de computación visual, el robot es capaz de reconocer el estado de ánimo del cachorro y adaptar su comportamiento, cambiando entre diferentes modos de juego y compañía. Adicionalmente, el robot es programable y en base a la edad del cachorro adaptar un comportamiento más materno. También es capaz de almacenar un mapa virtual de la casa y mantener al animal en las habitaciones que desee el dueño.


\subsection{Entregable 3}
% – Haga la prueba de Hersey-Blanchard y comente su resultado.

\subsubsection{Herser-Blanchard test}

Mis respuestas han sido: 1. A 2. A 3. A 4. D 5. C 6. D 7. B 8. A 9. C 10. D 11. B 12. D

Dándome como resultado en amplitud de estilos:
\begin{itemize}
    \item Directivo: 3
    \item Persuasivo: 5
    \item Participativo: 3
    \item Delegativo: 1
\end{itemize}

Y en adaptabilidad de estilos:
\begin{itemize}
    \item Directivo: 0
    \item Persuasivo: -1
    \item Participativo: 2
    \item Delegativo: 18
    \item \textbf{Suma}: 19
\end{itemize}

Siendo más altos de lo que me esperaba en persuasivo, pero me cuadra en participativo. Siento que mi perspectiva es ``si algo funciona bien, mejor dejárlo'' e intentar involucrar a todos y que se sientan contentos (yo incluído). El alto valor en directivo lo achaco a que aunque intentaría ser flexible y amigable, siendo líder soy responsable de mis subordinados, y por ello dependo de sus buenos resultados.

Sobre la adaptabilidad, los resultados me concuerdan ya que considero que cada uno es ``experto'' en lo suyo y por tanto su opinión es tan válida como la del líder. Es este último el que debe escuchar, pero en última instancia valorar y decidir.

\subsubsection{Liderazgo situacional}

% – Escriba una reflexión sobre cómo ha aplicado o echado en falta la aplicación de un liderazgo situacional en su vida profesional (académica en su defecto). Nota: si usted no ha sido jefe/líder, escriba sobre su experiencia como subordinado.

Aunque no está dentro del ámbito profesional me gustaría hablar de mi experiencia personal como líder situacional en mi vida social. 
Para poner en situación, pertenezco a un grupo de 4 amigos con una buena costumbre de elegir cuidadosamente los regalos de cumpleaños. La cosa está en que soy el único del grupo preocupado por prever los regalos con antelación y no acabar entregándolos después de tiempo, por lo que suelo asumir el puesto de organizador todos los años.

\vspace{\baselineskip}

Uno de ellos es menos afán de regalos materiales, y siento un rol más participativo cuando propongo y coordino la realización de regalos hechos por nosotros, intentando que se note la participación y el cariño de cada uno.

El rol delegativo se ve con el segundo de ellos, puesto que al tener una amistad más afín con un amigo y menos con el resto del grupo suele indicarle que investigue sobre posibles ideas.

El liderazgo directivo no lo veo en este contexto, pero por el test de Herser-Blanchard tampoco parece que sea muy frecuente en mi persona.

% Mencionar los cuatro tipos de liderazgo

% Entrega 3, cuestionario en SWAD (hacerlo es opcional)
% La reflexión solo vale. Líder en lo que sea propia (amigos ? trabajos de universidad?)
% Sino subordinado identificando los 4 estilos de liderazgo (Identificarlo siempre)



% • Distribución
% • Servicio
% • Compra
% • Transporte
% • Formación
% • Garantías

% ¿Por qué mi producto o servicio es mejor?
% • ¿Cómo conseguir que me compren a mí?
% • ¿Cómo puedo sobrevivir a la competencia futura?

    % ==============================================================================

    \setlength{\parskip}{1em}
    \newpage
    % \nocite{*}
    % \bibliography{bibliografia}
  	% \bibliographystyle{plain}
\end{document}