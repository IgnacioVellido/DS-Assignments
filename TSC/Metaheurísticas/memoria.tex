\documentclass[13pt,a4paper]{article}
\usepackage[spanish,es-nodecimaldot]{babel}	% Utilizar español
\usepackage[utf8]{inputenc}					% Caracteres UTF-8
\usepackage{graphicx}						% Imagenes
\usepackage[hidelinks]{hyperref}			% Poner enlaces sin marcarlos en rojo
\usepackage{fancyhdr}						% Modificar encabezados y pies de pagina
\usepackage{float}							% Insertar figuras
\usepackage[textwidth=390pt]{geometry}		% Anchura de la pagina
\usepackage[nottoc]{tocbibind}				% Referencias (no incluir num pagina indice en Indice)
\usepackage{enumitem}						% Permitir enumerate con distintos simbolos
% \usepackage[T1]{fontenc}					% Usar textsc en sections
\usepackage{amsmath}						% Símbolos matemáticos
\usepackage[ruled,vlined]{algorithm2e}      % Pseudocódigo
\usepackage{xcolor}
\usepackage{listings}
% Para que acepten tíldes los listing
\lstset{     
     literate=%
         {á}{{\'a}}1
         {é}{{\'e}}1
         {í}{{\'i}}1
         {ó}{{\'o}}1
         {ú}{{\'u}}1
         {Á}{{\'A}}1
         {É}{{\'E}}1
         {Í}{{\'I}}1
         {Ó}{{\'O}}1 
         {Ú}{{\'U}}1
         {ñ}{{\~n}}1 
         {Ñ}{{\~N}}1 
         {¿}{{?``}}1 
         {¡}{{!``}}1
}
\usepackage{dsfont}
\usepackage{caption, subcaption}

% ==============================================================================

\usepackage{caption}
\usepackage[section]{placeins}
\makeatletter
\def\fps@figure{H}
\makeatother

\usepackage{booktabs}
\usepackage{longtable}
\usepackage{array}
\usepackage{multirow}
\usepackage{wrapfig}
\usepackage{colortbl}
\usepackage{pdflscape}
\usepackage{tabu}
\usepackage{threeparttable}
\usepackage{threeparttablex}
\usepackage[normalem]{ulem}
\usepackage{makecell}
\usepackage[bottom]{footmisc}

% ==============================================================================
% ==============================================================================

% Comando para poner el nombre de la asignatura
\newcommand{\asignatura}{TSCAO}
\newcommand{\autor}{Ignacio Vellido Expósito}
\newcommand{\email}{ignaciove@correo.ugr.es}
\newcommand{\titulo}{Metaheurísticas}
\newcommand{\subtitulo}{Trabajo final}

% Configuracion de encabezados y pies de pagina
\pagestyle{fancy}
\lhead{\autor{}}
\rhead{\asignatura{}}
\lfoot{Máster Ciencia de Datos e Ingeniería de Computadores}
\cfoot{}
\rfoot{\thepage}
\renewcommand{\headrulewidth}{0.4pt}		% Linea cabeza de pagina
\renewcommand{\footrulewidth}{0.4pt}		% Linea pie de pagina

% ==============================================================================
% ==============================================================================

\begin{document}
    \pagenumbering{gobble}
    % ==============================================================================
% Pagina de titulo
\begin{titlepage}
    \begin{minipage}{\textwidth}
        \centering

        \includegraphics[scale=0.5]{img/ugr.png}\\

        \textsc{\Large \asignatura{}\\[0.2cm]}
        \textsc{MÁSTER CIENCIA DE DATOS E INGENIERÍA DE COMPUTADORES}\\[1cm]

        \noindent\rule[-1ex]{\textwidth}{1pt}\\[1.5ex]
        \textsc{{\Huge \titulo\\[0.5ex]}}
        \textsc{{\Large \subtitulo\\}}
        \noindent\rule[-1ex]{\textwidth}{2pt}\\[2.5ex]

        \end{minipage}

        \vspace{0.3cm}

        \begin{minipage}{\textwidth}

        \centering

        \textbf{Autor}\\ {\autor{}}\\[1.5ex]
        \vspace{0.4cm}

        \includegraphics[scale=0.3]{img/etsiit.jpeg}
        \includegraphics[scale=0.6]{img/master.png}

        \vspace{0.7cm}
        \textsc{Escuela Técnica Superior de Ingenierías Informática y de Telecomunicación}\\
        \vspace{1cm}
        \textsc{Curso 2020-2021}
    \end{minipage}
\end{titlepage}
% ==============================================================================
    
    \pagenumbering{arabic}    
    \newpage

    % ==============================================================================

\section{Maximum Diversity Problem (MD)}

\subsection{Búsqueda bibliográfica}

% https://www.scopus.com/results/results.uri?editSaveSearch=&sort=cp-f&src=s&nlo=&nlr=&nls=&sid=0e91577fead34b1c9e4d6e2f33578357&sot=b&sdt=sisr&cluster=scosubjabbr%2c%22COMP%22%2ct&sl=40&s=TITLE-ABS-KEY%28maximum+diversity+problem%29&ref=%28mdp%29&origin=resultslist&zone=leftSideBar&txGid=c96d17e4d26110d2035d852c0e3d93cc

\begin{itemize}
    \item Lopez-Pires, Fabio \& Vera, Katherine \& Baran, Benjamin \& Sandoya, Fernando. (2017). Multi-Objective Maximum Diversity Problem. 10.1109/CLEI.2017.8226423.
    \item Marti, Rafael \& Gallego, Micael \& Duarte, Abraham. (2010). A branch and bound algorithm for the maximum diversity problem. European Journal of Operational Research. 200. 36-44. 10.1016/j.ejor.2008.12.023. 
    \item Aringhieri, Roberto \& Cordone, Roberto. (2008). Tabu Search versus GRASP for the maximum diversity problem. 4OR. 6. 10.1007/s10288-007-0033-9. 
    \item Parreño, Francisco \& ÁLvarez-Valdés, Ramón \& Marti, Rafael. (2020). Measuring Diversity. A review and an empirical analysis. European Journal of Operational Research. 289. 10.1016/j.ejor.2020.07.053. 
    \item Marti, Rafael \& Martínez-Gavara, Anna \& Sánchez-Oro, Jesús. (2021). The capacitated dispersion problem: an optimization model and a memetic algorithm. Memetic Computing. 13. 10.1007/s12293-020-00318-1. 
    \item Marti, Rafael \& Gallego, Micael \& Duarte, Abraham \& G. Pardo, Eduardo. (2013). Heuristics and metaheuristics for the maximum diversity problem. Journal of Heuristics - HEURISTICS. 19. 1-25. 10.1007/s10732-011-9172-4. 
    \item Silva, Geiza \& Ochi, Luiz \& Martins, Simone. (2004). Experimental Comparison of Greedy Randomized Adaptive Search Procedures for the Maximum Diversity Problem. Lecture Notes on Computer Science. 3059. 498-512. 10.1007/978-3-540-24838-5\_37. 
    \item Zhou, Yangming \& Hao, Jin-Kao \& Duval, Beatrice. (2017). Opposition-Based Memetic Search for the Maximum Diversity Problem. IEEE Transactions on Evolutionary Computation. 21. 731-745. 10.1109/TEVC.2017.2674800. 
    \item Gallego, Micael \& Duarte, Abraham \& Laguna, Manuel \& Marti, Rafael. (2009). Hybrid heuristics for the maximum diversity problem. Computational Optimization and Applications. 44. 411-426. 10.1007/s10589-007-9161-6. 
    \item Silva, Geiza \& Andrade, Marcos \& Ochi, Luiz \& Martins, Simone \& Plastino, Alexandre. (2007). New heuristics for the maximum diversity problem. J. Heuristics. 13. 315-336. 10.1007/s10732-007-9010-x. 
    \item Santos, L. \& Ribeiro, Marcos \& Plastino, Alexandre \& Martins, Simone. (2005). A Hybrid GRASP with Data Mining for the Maximum Diversity Problem. Lecture Notes in Computer Science. 3636. 116-127. 10.1007/11546245\_11. 
    \item Zhou, Yalan \& Yin, Jian \& Zhang, Yunong. (2009). Competitive Hopfield Network Combined With Estimation of Distribution for Maximum Diversity Problems. Systems, Man, and Cybernetics, Part B: Cybernetics, IEEE Transactions on. 39. 1048 - 1066. 10.1109/TSMCB.2008.2010220. 
    \item Andrade, Marcos \& Andrade, Paulo \& Martins, Simone \& Plastino, Alexandre. (2005). GRASP with Path-Relinking for the Maximum Diversity Problem. Lecture Notes in Computer Science. 3503. 558-569. 10.1007/11427186\_48. 
    \item Lozano, Manuel \& Molina, Daniel \& GarcI´a-MartI´nez, C.. (2011). Iterated greedy for the maximum diversity problem. European Journal of Operational Research. 214. 31-38. 10.1016/j.ejor.2011.04.018. 
\end{itemize}

\subsection{Pseudocódigos}

\subsubsection{Greedy}

Un algoritmo greedy se define de la siguiente forma:

\begin{algorithm}[H]
    \SetAlgoLined
    \KwIn{datos: Conjunto de datos}
        sol = solución actual vacía \;
        \Repeat{hasta que sol sea un solución} {
            sol += elegirCandidatoGreedy(sol, datos) \;
        }        
    \Return{sol}
    \caption{Pseudocódigo algoritmo greedy}
\end{algorithm}

\vspace{\baselineskip}

Pudiendo ser \textbf{elegirCandidatoGreedy}:

\begin{algorithm}[H]
    \SetAlgoLined
    \KwIn{sol, datos}        
    \Return{candidato no presente en sol con mayor distancia a los elementos en sol}
    \caption{elegirCandidatoGreedy}
\end{algorithm}

% ---------------------------------------------------------------

\subsubsection{Semi-Greedy}

Un algoritmo semi-greedy se define de la siguiente forma:

\begin{algorithm}[H]
    \SetAlgoLined
    \KwIn{datos: Conjunto de datos}
        sol = solución actual vacía \;
        \Repeat{hasta que sol sea un solución} {
            candidatos = elegirMejoresCandidatosGreedy(sol, datos) \;
            sol += random(candidatos) \;
        }        
    \Return{sol}
    \caption{Pseudocódigo algoritmo semi-greedy}
\end{algorithm}

\vspace{\baselineskip}

Pudiendo ser \textbf{elegirMejoresCandidatosGreedy}:

\begin{algorithm}[H]
    \SetAlgoLined
    \KwIn{sol, datos}        
    \Return{N candidatos no presentes en sol ordenados por mayor distancia a sol}
    \caption{elegirMejoresCandidatosGreedy}
\end{algorithm}

% ---------------------------------------------------------------

\subsubsection{Iterated-Greedy}

Un algoritmo de iterated-greedy se define de la siguiente forma:

\begin{algorithm}[H]
    \SetAlgoLined
    \KwIn{datos: Conjunto de datos}
        sol = solución actual vacía \;
        \Repeat{hasta que sol cumpla los criterios de parada} {
            xd = destrucción(sol) \;
            xc = construcción(xd, datos) \;
            sol = aceptar(sol, xc) \;
        }        
    \Return{sol}
    \caption{Pseudocódigo algoritmo iterated-greedy}
\end{algorithm}

\vspace{\baselineskip}

Pudiendo ser \textbf{destrucción}:

\begin{algorithm}[H]
    \SetAlgoLined
    \KwIn{sol}
        sol = quitar N elementos aleatorios de la solución sol
    \Return{sol}
    \caption{destrucción}
\end{algorithm}

\vspace{\baselineskip}

Pudiendo ser \textbf{construcción}:

\begin{algorithm}[H]
    \SetAlgoLined
    \KwIn{sol, datos}        
    \Return{candidato no presente en sol con mayor distancia a los elementos en sol}
    \caption{construcción}
\end{algorithm}

\vspace{\baselineskip}

Pudiendo ser \textbf{aceptar}:

\begin{algorithm}[H]
    \SetAlgoLined
    \KwIn{sol, temp}
        best = sol \;
        \uIf{diversidad(temp) $>$ diversidad(sol)}{
            best = temp \;
        }         
    \Return{best}
    \caption{aceptar}
\end{algorithm}

% ---------------------------------------------------------------

\subsection{Búsqueda local}

\begin{algorithm}[H]
    \SetAlgoLined
    \KwIn{datos: Conjunto de datos}
        sol = solución aleatoria válida \;
        candidatos = vecindario(sol) \;
        \Repeat{hasta que candidatos esté vacía} {
            sol = best(candidatos) \;
            candidatos = vecindario(sol) \;
        }        
    \Return{sol}
    \caption{Pseudocódigo algoritmo de búsqueda local}
\end{algorithm}

\vspace{\baselineskip}

Pudiendo ser un posible operador de vecindario:

\begin{algorithm}[H]
    \SetAlgoLined
    \KwIn{sol, datos}        
    \Return{sustituir elemento i de sol}
    \caption{Operador de vecindario}
\end{algorithm}

% ---------------------------------------------------------------

\subsection{Algoritmo genético}

% Representación (cromosomas)
% Operador mutación
% Operador cruce
% Inicialización de población

Una posible representación podría ser un vector binario $x = (x_{0}, ..., x_{n})$ indicando si un elemento $x_{i}$ está presente o no en el subconjunto solución.

\vspace{\baselineskip}

De esta forma tendríamos como operadores de cruce y mutación:

\begin{algorithm}[H]
    \SetAlgoLined
    \KwIn{sol1, sol2}
        sol = solución vacía \;
        first = seleccionar $n/2$ elementos incluídos en sol1 \;
        second = seleccionar $n/2$ elementos incluídos en sol2 \;
        sol = first + second \;        
        \tcc{Reparar}
        sol = añadir o quitar elementos aleatoriamente hasta que sol sea válida \;
    \Return{sol}
    \caption{Operador de cruce}
\end{algorithm}

\vspace{\baselineskip}

\begin{algorithm}[H]
    \SetAlgoLined
    \KwIn{sol}
        sol = quitar $n$ elementos de sol aleatoriamente y añadir $n$ aleatoriamente \;
    \Return{sol}
    \caption{Operador de mutación}
\end{algorithm}

\vspace{\baselineskip}

Y una inicialización correspondería a seleccionar $m$ elementos de $x$ de forma aleatoria.

% ==============================================================================================
\newpage

\section{Multidimensional two-way number partitioning problem (M2NP)}

\subsection{Búsqueda bibliográfica}

% https://www.scopus.com/results/results.uri?src=s&st1=&st2=&sot=b&sdt=b&origin=searchbasic&rr=&sl=59&s=TITLE-ABS-KEY(multidimensional%20two%20way%20number%20partitioning)&searchterm1=multidimensional%20two%20way%20number%20partitioning&searchTerms=&connectors=

\begin{itemize}
    \item Kojić, Jelena. (2010). Integer linear programming model for multidimensional two-way number partitioning problem. Computers \& Mathematics with Applications. 60. 2302-2308. 10.1016/j.camwa.2010.08.024. 
    \item Alexandre Frias Faria, Sérgio Ricardo de Souza, Elisangela Martins de Sá, A mixed-integer linear programming model to solve the Multidimensional Multi-Way Number Partitioning Problem, Computers \& Operations Research, Volume 127, 2021, 105133, ISSN 0305-0548.
    \item Santucci, Valentino \& Baioletti, Marco \& Di Bari, Gabriele \& Milani, Alfredo. (2019). A Binary Algebraic Differential Evolution for the MultiDimensional Two-Way Number Partitioning Problem. 10.1007/978-3-030-16711-0\_2. 
    \item Hacibeyoglu, Mehmet \& Alaykiran, Kemal \& ACILAR, A. Merve \& Tongur, Vahit \& Ülker, Erkan. (2018). A Comparative Analysis of Metaheuristic Approaches for Multidimensional Two-Way Number Partitioning Problem. Arabian Journal for Science and Engineering. 43. 10.1007/s13369-018-3155-9. 
    \item Jozef Kratica, Jelena Kojić, Aleksandar Savić,Two metaheuristic approaches for solving multidimensional two-way number partitioning problem,Computers \& Operations Research,Volume 46,2014,Pages 59-68,ISSN 0305-0548,
    \item Pop, Petrica \& Matei, Oliviu. (2013). A Genetic Algorithm Approach for the Multidimensional Two-Way Number Partitioning Problem. 7997. 81-86. 10.1007/978-3-642-44973-4\_10. 
    \item Petrică C. Pop, Oliviu Matei,A memetic algorithm approach for solving the multidimensional multi-way number partitioning problem,Applied Mathematical Modelling,Volume 37, Issue 22,2013,Pages 9191-9202,ISSN 0307-904X,
    \item Vera, J. \& Macías, Rodrigo \& Heiser, Willem. (2009). A Latent Class Multidimensional Scaling Model for Two-Way One-Mode Continuous Rating Dissimilarity Data. Psychometrika. 74. 297-315. 10.1007/s11336-008-9104-x.
\end{itemize}

% ==========================================================================================

% La función objetivo en el problema PAR se calcula en base a la fórmula:

% \begin{equation}
%     f = \overline{C} + (infeasibility * \lambda)
% \end{equation}

% Siendo:

% \begin{equation}
%     \lambda = \frac{\max \{d_i \in D\}}{|R|} \quad tal\ que \ D = Distancias
% \end{equation}

% \begin{equation}
%     infeasibility = \sum_{i=0}^{|ML|} \mathds{1}(h_{C}(\overrightarrow{ML_{[i,1]}}) \neq (h_{C}(\overrightarrow{ML_{[i,2]}}) + \sum_{i=0}^{|CL|} \mathds{1}(h_{C}(\overrightarrow{CL_{[i,1]}}) = (h_{C}(\overrightarrow{CL_{[i,2]}})
% \end{equation}

% \begin{equation}
%     \overline{C} = \frac{1}{k} \sum_{c_{i}\in C} || \overrightarrow{x_j} - \overrightarrow{u_j} ||_{2}
% \end{equation}

% Que es calculada en pseudocódigo: \\

% \begin{algorithm}[H]
%     \SetAlgoLined
%         C $=$ Distancia media intra-cluster \;
%         $\lambda =$ Distancia máxima en el conjunto de datos / nº de restricciones  \;
%         inf $=$ Nº restricciones no cumplidas \;
%     \Return{C + ($\lambda *$ inf)}
%     \caption{Función objetivo}
% \end{algorithm}

% \vspace{\baselineskip}

% \begin{algorithm}[H]
%     \SetAlgoLined
%     \KwIn{Conjunto de datos, solucion, centroides}
%         Separar conjunto de datos según su cluster \;
%         \For{particion del conjunto de datos} {
%             Calcular distancia media de sus elementos al centroide correspondiente \;
%         }        
%     \Return{C}
%     \caption{Distancia media intra-cluster}
% \end{algorithm}

% \vspace{\baselineskip}

% \begin{algorithm}[H]
%     \SetAlgoLined
%     \SetKw{KwInFor}{in}
%     \KwIn{s: solución}
%         inf $=$ 0 \;
%         \For{r \textbf{in} $lista\_restricciones$} {
%             \uIf{r $=$ ML \textbf{and} $s[r[0]] \neq s[r[1]]$} {
%                 inf++ \;
%             }            
%             \uElseIf{r $=$ CL \textbf{and} $s[r[0]] = s[r[1]]$} {
%                     inf++ \;
%             }
%         }
%     \Return{inf}
%     \caption{Infeasibility}
% \end{algorithm}

% =============================================================================================

% MGS
% https://www.scopus.com/results/results.uri?cc=10&sort=r-f&src=s&nlo=&nlr=&nls=&sid=7659d39a8257beebb53de9c8729fa21d&sot=b&sdt=cl&cluster=scosubjabbr%2c%22COMP%22%2ct&sl=37&s=TITLE-ABS-KEY%28minimum+generating+set%29&ss=r-f&ps=r-f&editSaveSearch=&origin=resultslist&zone=resultslist

% A genetic algorithm for the minimum generating set problem

% The Complexity of Generating Minimum Test Sets for PLA's and Monotone Combinational Circuits

% The complexity of quasigroup isomorphism and the minimum generating set problem

    % ==============================================================================
    \setlength{\parskip}{1em}
    \newpage
\end{document}

% ==============================================================================
% ==============================================================================